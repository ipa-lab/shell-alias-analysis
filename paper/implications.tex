\section{Implications}

Through our large-scale analysis 
%of unique files that contain over 2.2 million alias definitions, 
we gained insight into practices detailing how users customize their command-line interface.
Based on our observations, we outline discussion points that go beyond single customization practices and identify implications that can address shortcomings in command-line user experience.
Further, while our presented findings already give us an understanding of customization practices over many different kinds of commands, we view our collected dataset as a playground for fine-grained discovery that can benefit researchers, tool builders, and command-line users.

\paragraph{\bf Learning Repair Rules}

The complexity of commands and arguments can cause users to introduce errors when working in a command line interface.
Figuring out specifically how to fix these errors is often a convoluted process.
There is a popular open source project that attempts to navigate this issue.\footnote{\url{https://github.com/nvbn/thefuck}}
It uses a set of rules to suggest possible error corrections to the command.
While these rules are all hard-coded, we envision leveraging the global wisdom of customizations in our large scale dataset to learn rules that form the basis for different kinds of suggestions.
%An obvious choice from our analysis are autocorrecting aliases to learn repair rules.
\paragraph{\bf Discovering Workflows}
We can also see how our dataset would enable a world beyond only trying to fix immediate errors, but also provide usage hints that could introduce users to common parameters and workflows overall.
\TODO example for workflow
%\paragraph{\bf Object Protocols}
Similar to workflows, we have also seen the prevalence of object protocols~\cite{beckman:11}, which are implicit rules determining  the order in which commands have to be executed.
We can improve usability by enabling the discovery of these rules and exposing the dependency structure from our customization data.
For instance, if executing \verb|brew upgrade| results in a failure, we can suggest using \verb|brew update && brew upgrade| instead based on the customization patterns we found on chaining subcommands.

\paragraph{\bf Uncovering Conceptual Design Flaws}

Customization can also be an indicator for problems in the underlying conceptual design that manifest as usability frustrations that require adaption.
e want to briefly compare to the qualitative analysis on conceptual design flaws in \verb|git| by Perez and Jackson~\cite{perez:13}.
Particularly, we found the frustrations with committing and switching branches\footnote{``Just Let Me Commit!" and ``I Just Want to Switch Branches" in the paper~\cite{perez:13}} presented as in many customizations in our dataset.
\TODO list the particular connections from our data to the paper

We want to emphasize that we are not suggesting that data on large-scale customization practice can replace qualitative analysis, but rather that the corpus we provide insights for exploration and usability research.

\paragraph{\bf Contextual Defaults}

Choosing proper defaults in user interfaces is a pillar of user experience design~\cite{nielsen2005power}.
We see approximately 14\% (XXX) of our customizations are overriding default use in various ways.
%This practice is invasive. 
It means \emph{all} uses of this command are overwritten to adhere to the customization. 
%We can see different interpretations for this practice.
We see overriding defaults as an indication that the assumed default does not match the usage profile.
The variety of different defaults in the data indicate what we call contextual defaults, where context could be the expertise level of a command-line user or a certain persona (e.g., sysadmin, data scientist, software engineer).
%Different kinds of users require different default entry points from a command.

We could imagine providing a different set of aliases to different users (e.g., alias starter packs for different levels of expertise) generated from our data.

%We  a particular argument of a command (or set of arguments) repeatedly show up in customizations where that command is redefined as a new default,  
%it could be an indicator for the repeated argument structure to become a default.



\paragraph{\bf Interactive Shell vs Batch-Processing}

Context can be also seen as the environment a command is executed in. 
There is a difference in use of commands as part of scripting for batch-processing, compared to interactive use on the command-line.
Our findings highlight the tension between commands used in scripting for automation, and as part of an interactive medium to interface with the operating system.
This particularly comes to light when highlighting our findings on customizations that redefine commands with defaults for interactive safety:
While it makes no sense to use the \verb|--interactive| or \verb|-i| argument, our findings show that it is a highly desired customization practice for interactive command-line use.
