\section{Conclusion}

We report on a large scale exploratory study on how command-line users customize user experience by defining shell aliases.
Through inductive coding, nine customization practices emerged from our dataset of collective customization knowledge mined from GitHub, providing insight on the characteristics of command-line use.
Based on our results, we discuss and formulate a set of implications for command-line tool developers, researchers, and the shell as an interactive environment for experts.
We enable further analysis and a basis for learning applications based on our extensive curated dataset.

Aliases often redefine commands with other default arguments, which is a potential indicator for usability problems in these tools.
However, we have to also be aware that defaults can be highly contextual depending on user profiles (e.g., expertise level) and environment (e.g., scripting vs. interactive use).
We also see our dataset and results as a rich source for learning norms with respect to repair rules, data flows, and descriptive names for complex command structures.
We provide a comprehensive replication package and see potential for future work based on our dataset and analyses.