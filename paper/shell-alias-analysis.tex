\RequirePackage{fix-cm}
\documentclass[smallextended,natbib]{svjour3}
\usepackage[T1]{fontenc}
\usepackage[utf8]{inputenc}
\usepackage{url}
\usepackage[colorlinks,linkcolor=blue,citecolor=blue,urlcolor=blue]{hyperref}
\usepackage[autolanguage]{numprint}
\usepackage{booktabs}
\usepackage{graphicx}
\usepackage{wasysym}
\usepackage{fancyvrb}
\usepackage{fvextra}
\usepackage{setspace}
\usepackage{tikz}
\usepackage{cprotect}
\usepackage{float}
\usepackage{adjustbox}
\usepackage{enumitem}

\usepackage{xspace}
\newcommand*{\TODO}{\textcolor{red}{TODO}\xspace}

\newcommand{\num}[1]{\numprint{#1}}
\newcommand{\per}[1]{\numprint[\%]{#1}}

\newenvironment{CVerbatim}
  {\center\BVerbatim}
  {\endBVerbatim\endcenter}

\newcommand{\alias}[2]{{\texttt{#1} $\rightarrow$ \texttt{#2}}}
\newcommand{\cmd}[1]{{\texttt{#1}}}

\journalname{Empirical Software Engineering}

\begin{document}

\title{An Empirical Study of Command-Line Customization}

\author{Michael Schröder \and Jürgen Cito}
\institute{%
	Michael Schröder
	\at TU Wien, Vienna, Austria\\
	\email{michael.schroeder@tuwien.ac.at}
	\and
	Jürgen Cito
	\at TU Wien \& MIT, Cambridge, MA\\
	\email{jcito@mit.edu}
}

\maketitle

\begin{abstract}
	The interactive command-line is a prominent mechanism that a wide range of software professionals (engineers, sysadmins, data scientists, etc.) use extensively during development and other systems activities. Shell command customizations can therefore provide insight into the tasks that they repeatedly perform, how well the standard environment supports those tasks, and ways in which this standard environment could be productively extended or modified.
To characterize the patterns and complexities of command-line customization, we conducted an exploratory empirical study on over 2.2 million alias definitions found on GitHub.
Shell aliases allow command line users to customize their environment by defining (arbitrarily complex) command substitutions.
Using inductive coding methods, we found that there are three types of aliases and that each enables a number of complex customization scenarios: 
\textsc{Shortcuts} (for \emph{nicknaming commands}, \emph{abbreviating subcommands}, and \emph{bookmarking locations}),
\textsc{Modifications} (for \emph{substituting commands}, \emph{overriding defaults}, \emph{colorizing output}, and \emph{elevating privilege}),
and \textsc{Scripts} (for \emph{transforming data} and \emph{chaining subcommands}).
We conjecture that identifying common customization practices can point to particular usability issues within command-line programs, and that a deeper understanding of these practices can support researchers and tool developers in designing better user experiences.
	\keywords{Command Line \and Customization Practices \and Collective Knowledge \and Inductive Coding}
\end{abstract}

\section{Introduction}

A command line interface, also called a \emph{shell}, is a textual interface that allows users to interact with the underlying operating system by issuing commands.
Expert users, such as system administrators, software developers, researchers, and data scientists, routinely use the shell as it affords them flexibility and the ability to compose multiple commands.
They perform a variety of tasks on their systems including navigating and interacting with the filesystem (e.g., \verb|ls|, \verb|mv|, \verb|cd|), using version control (e.g., \verb|git|, \verb|hg|), installing packages (e.g., \verb|apt-get|, \verb|npm|), or dealing with infrastructure (e.g., \verb|docker|).
Experts can adapt and play with a multitude of commands and arguments, chaining them together to create more complex workflows.
All this versatility introduces a common problem in user interfaces of recognition over recall~\cite{nielsen:05}, where users have to recall the particularities of syntax and argument combinations, instead of enabling them to use a more recognizable symbol (as in graphical user interfaces).

A way for these experts to introduce recognizability and customize their command line experience is to attach a distinct name to potentially convoluted, but frequently used, command and argument structures.
This can be achieved by defining shell aliases.
An alias substitutes a given name, the \emph{alias}, with a string value that defines an arbitrarily complex command (or chain of commands).
The set of aliases users define provides a window into their preferences expressed as part of their personal shell configuration.
Many users have publicly shared these configurations on social coding platforms such as GitHub.
To study these preferences, we performed an exploratory analysis of over 2.2 million alias definitions on GitHub, which provides a rich dataset to investigate how command line users customize their tasks and workflows.

We find that there are three types of aliases, supporting a number of customization practices:
\textsc{Shortcuts} introduce new names.
They can be used for \emph{nicknaming commands} (and correcting misspellings in the process),
\emph{abbreviating subcommands} like \texttt{git push},
and \emph{bookmarking locations} for quick navigation.
\textsc{Modifications} change the semantics of commands.
We can use these types of aliases for \emph{substituting commands}, such as replacing \cmd{more} with \cmd{less},
for \emph{overriding defaults} to customize commands to personal contexts, 
which often involves \emph{colorizing output},
and also running certain commands as root by \emph{elevating privilege}.
Aliases that combine multiple commands are \textsc{Scripts}.
They enable many ways of \emph{transforming data} using Unix pipes, 
and allow for automating repetitive workflows by \emph{chaining subcommands}.

\TODO Our insights can help improve the usability for particular tools and the command line experience in general.
Our results on default arguments indicate usability issues for a wide range of tools, but also shows that defaults can sometimes be contextual (e.g., \verb|-i| interactive mode by default in the command line, but not when executing scripts).
We also envision that our results and dataset can be used to learn repair rules, in order to correct not only user errors, but also provide usage hints to introduce newcomers to common workflows.

In the following, we introduce more background by describing usage and syntax of aliases. We further describe our data collection and parsing process. We then present the findings from our exploratory analysis on shell aliases in the wild, followed by a discussion on implications for usability. We finally discuss threats to validity and review related work in the broader context of this study. 

\section{Background}

A shell is a command interpreter allowing the user to interact with an underlying system.
The concept of the operating system shell as an independent process executing outside the kernel originated in Multics and was further developed into the original Unix shell \verb|sh| and its various descendants.
%  (\TODO references)
The POSIX family of standards defines a Shell Command Language~\cite{posix_standard, greenberg:17}, whose standard implementation is still the \verb|sh| utility, but there exist a wide variety of popular POSIX-compliant shells like \verb|bash| or \verb|zsh|.
These implementations are free to extend the functionality of the shell, but all share a common subset of core commands and programming language constructs.
In this paper, we focus on the built-in \verb|alias| command, available on all POSIX shells.

\subsection{Usage and Syntax}

The \verb|alias| command allows the user to create \emph{alias definitions}, defining command substitutions.
When the shell processes the command line, it replaces known alias names with their defined string values.
For example, 
\begin{CVerbatim}
alias ll='ls -l'
\end{CVerbatim}
defines the \emph{alias name} \verb|ll|, that is replaced by the \emph{alias value} \verb|ls -l|.
In this case, \verb|ls| is the standard command for listing directory contents, with the argument \verb|-l| specifying a long-form output format.
So the alias \verb|ll| (present in many system configurations) is used to specify a default argument to a commonly used command under a different name.

Alias values can be arbitrarily complex strings and can substitute not only simple commands and arguments, but whole chains of commands. 
The definition
\begin{CVerbatim}
alias ducks="du -cksh * | sort -hr | head -n 15"
\end{CVerbatim}
defines the new command \verb|ducks| by chaining together three different command line tools in order to return the 15 largest files in the current directory.

%alias ip=ifconfig | grep "inet " | grep -v 127.0.0.1 | cut -d' ' -f2
%which chains together several commands to find out the IP addresses of the system.

%alias unwip="git log -n 1 | grep -q -c wip && git reset HEAD~1"
%which chains together three commands to rewind the most recent commit in a git repository, if its commit message contains the word "wip".

In general, an alias definition takes the form
\begin{CVerbatim}
alias name=value
\end{CVerbatim}
where \verb|value| can optionally be enclosed in single (\verb|'|) or double (\verb|"|) quotes and \verb|name| can be any identifier that is a valid command name.\footnote{Some shells allow for an alternative alias syntax without the equals sign between \texttt{name} and \texttt{value}. In this paper we only look at POSIX-compliant alias definitions.}
In particular, the alias name can be an existing command, so a re-definition like
\begin{CVerbatim}
alias grep='grep --color=always'
\end{CVerbatim}
is possible.

\subsection{Dotfiles}

Aliases can be entered directly on the command line, in which case they are valid until the shell session ends.
To make an alias definition permanent, it is common practice to enter it into a file that is read and executed by the shell on startup.
The names of these shell configuration files differ by platform, but common ones are \verb|.bashrc|, \verb|.zshrc| or \verb|.profile|.
Often, aliases are also stored in other files referred to by these startup scripts.

These kinds of files --- text-based configuration files that store system or application settings --- are also known as \emph{dotfiles}, because their filenames usually start with a dot (\verb|.|) so that they are hidden by default on most Unix-based systems.
In recent years, people have started sharing their dotfiles on platforms like GitHub.\footnote{\url{https://dotfiles.github.io}}
This has the advantage of being able to sync one's configurations across different machines, and also enables exchange and discovery of configurations between users.


\section{Dataset}

\subsection{Data Collection}

TODO: describe github-searcher tool

\subsection{Structure}
TODO: describe decomposition, parser, database schema
TODO: basic stats about dataset: number of files, size distribution, where they originate (dotfiles etc; grep repo description and file path)

\subsection{Reproducibility}

TODO: describe steps to ensure reproducibility, i.e., sharing all scripts, collected data, and so on


\section{Shell Aliases in the Wild}

The single most common alias definition in our dataset is
\begin{verbatim}
alias '${1+"$@"}'="$@"
\end{verbatim}
appearing \numprint{78265} times, which is \numprint[\%]{1.64} of all alias definitions.
This curious jumble of symbols derives from a portability hack for \texttt{zsh}.\footnote{\url{https://unix.stackexchange.com/a/191836}}

More familiar results can be seen in \Cref{tab:top-summary}, which shows the most common alias names, commands, and arguments appearing in alias definitions.
The most common alias name we found is \texttt{ls}, appearing a total number of \numprint{124515} times, which is \numprint[\%]{2.61} of all alias definitions.
Note that this is \texttt{ls} as an \emph{alias name}, a redefinition of the \texttt{ls} \emph{command}, which appears \numprint{378097} times (\numprint[\%]{7.12}).
This is about half as often as \texttt{git}, the most common command, which appears in \numprint{626185} aliases (\numprint[\%]{11.79}).
The most common argument, across all commands, is \texttt{--color=auto}, appearing \numprint{268171} times (\numprint[\%]{3.6})

\begin{table*}
    \caption{Top alias names, commands and arguments}
    \label{tab:top-summary}
    \begin{tabular}{lrr}
        \toprule
                 Alias Name &                 \# &               \% \\
        \midrule
                  \verb|ls| &  \numprint{124515} &  \numprint{2.61} \\
                  \verb|ll| &   \numprint{89254} &  \numprint{1.87} \\
         \verb|'${1+"$@"}'| &   \numprint{78265} &  \numprint{1.64} \\
                \verb|grep| &   \numprint{63392} &  \numprint{1.33} \\
                  \verb|la| &   \numprint{62039} &  \numprint{1.30} \\
        \bottomrule
    \end{tabular}
    \hspace{0.1cm}
    \begin{tabular}{lrr}
        \toprule
               Command &                 \# &                \% \\
        \midrule
            \verb|git| &  \numprint{626185} &  \numprint{11.79} \\
             \verb|ls| &  \numprint{378097} &   \numprint{7.12} \\
             \verb|cd| &  \numprint{229522} &   \numprint{4.32} \\
           \verb|grep| &  \numprint{128045} &   \numprint{2.41} \\
           \verb|"$@"| &   \numprint{78276} &   \numprint{1.47} \\
        \bottomrule
    \end{tabular}
    \hspace{0.1cm}
    \begin{tabular}{lrr}
        \toprule
                    Argument &                 \# &               \% \\
        \midrule
         \verb|--color=auto| &  \numprint{268171} &  \numprint{3.60} \\
                   \verb|-i| &  \numprint{109094} &  \numprint{1.47} \\
                   \verb|-v| &   \numprint{70424} &  \numprint{0.95} \\
                   \verb|-l| &   \numprint{63302} &  \numprint{0.85} \\
                   \verb|-a| &   \numprint{62023} &  \numprint{0.83} \\
        \bottomrule
    \end{tabular}        
\end{table*}

Looking at each part of an alias definition in isolation can only get us so far, as arguments only gain meaning in conjunction with commands and alias names potentially vary widely.
\Cref{tab:command-summary} gives a more informative view, centered around some common commands, showing us the top arguments given with each command and the most common alias names by which the command/argument combinations are referred to.
Here we can see \TODO

\begin{table*}
    \caption{Common commands and their top arguments and aliases}
    \label{tab:command-summary}
    \begin{tabular}{lrll}
        \toprule
             Command &                \% &                 Arguments &                                                                                 Aliases (\%) \\
        \midrule
          \verb|git| &   \numprint{3.75} &             \verb|status| &                                  \verb|gs| (\numprint{51.57}), \verb|gst| (\numprint{24.17}) \\
                     &   \numprint{2.42} &           \verb|checkout| & \verb|gco| (\numprint{51.17}), \verb|gc| (\numprint{10.04}), \verb|letcat| (\numprint{9.18}) \\
                     &   \numprint{2.33} &               \verb|push| &    \verb|gp| (\numprint{49.12}), \verb|rulz| (\numprint{9.53}), \verb|gps| (\numprint{7.15}) \\
                     &   \numprint{2.31} &                   \verb|| &                                                                  \verb|g| (\numprint{78.87}) \\
                     &   \numprint{2.18} &               \verb|pull| &    \verb|gl| (\numprint{25.32}), \verb|gpl| (\numprint{13.29}), \verb|gp| (\numprint{11.53}) \\
                     &   \numprint{2.11} &               \verb|diff| &                                                                 \verb|gd| (\numprint{83.21}) \\
                     &   \numprint{2.05} &                \verb|add| &                              \verb|ga| (\numprint{73.01}), \verb|chicken| (\numprint{10.85}) \\
                     &   \numprint{2.01} &             \verb|branch| &                                   \verb|gb| (\numprint{74.88}), \verb|gbr| (\numprint{8.26}) \\
                     &   \numprint{1.59} &          \verb|commit -m| & \verb|gcm| (\numprint{21.67}), \verb|gcmsg| (\numprint{19.59}), \verb|gc| (\numprint{18.18}) \\
                     &   \numprint{1.20} &             \verb|commit| &                                                                 \verb|gc| (\numprint{65.51}) \\
        \midrule
           \verb|ls| &  \numprint{13.15} &       \verb|--color=auto| &                                                                 \verb|ls| (\numprint{99.09}) \\
                     &   \numprint{9.12} &                 \verb|-A| &                                                                 \verb|la| (\numprint{98.27}) \\
                     &   \numprint{8.57} &                \verb|-CF| &                                                                  \verb|l| (\numprint{98.94}) \\
                     &   \numprint{6.69} &                 \verb|-l| &                                                                 \verb|ll| (\numprint{82.64}) \\
                     &   \numprint{6.63} &               \verb|-alF| &                                                                 \verb|ll| (\numprint{97.89}) \\
                     &   \numprint{3.65} &                   \verb|| &      \verb|l| (\numprint{21.38}), \verb|sl| (\numprint{16.99}), \verb|iz| (\numprint{12.24}) \\
                     &   \numprint{3.06} &                 \verb|-G| &                                                                 \verb|ls| (\numprint{97.46}) \\
                     &   \numprint{2.43} &                \verb|-la| &      \verb|ll| (\numprint{32.80}), \verb|la| (\numprint{22.61}), \verb|l| (\numprint{12.71}) \\
                     &   \numprint{2.01} &                 \verb|-a| &                                                                 \verb|la| (\numprint{74.80}) \\
                     &   \numprint{1.77} &               \verb|-lah| &     \verb|l| (\numprint{32.64}), \verb|lsa| (\numprint{31.61}), \verb|ll| (\numprint{19.01}) \\
        \midrule
         \verb|grep| &  \numprint{36.12} &       \verb|--color=auto| &                                                               \verb|grep| (\numprint{99.29}) \\
                     &  \numprint{12.50} &                   \verb|| &   \verb|G| (\numprint{22.27}), \verb|hgrep| (\numprint{10.98}), \verb|rrg| (\numprint{7.22}) \\
                     &   \numprint{4.42} &            \verb|--color| &                                                               \verb|grep| (\numprint{97.00}) \\
                     &   \numprint{1.68} &                 \verb|-i| &        \verb|grep| (\numprint{6.88}), \verb|G| (\numprint{5.86}), \verb|g| (\numprint{5.86}) \\
                     &   \numprint{1.52} & \verb|--color=never '^d'| &                                                                \verb|lsd| (\numprint{98.45}) \\
        \bottomrule
    \end{tabular}
\end{table*}

One drawback of this presentation is that there are obviously many different ways of how an arguments to a command can be combined, since for most programs, the order of arguments does not matter and often there are multiple ways to specify the same argument.
\TODO


\subsection{Default Arguments}

One common use case of using aliases was to define commands with default arguments.
More specifically, we refer to an alias as defining default arguments when the alias name is the same as the command name, and in the alias definition, the command specifies at least one further argument.
This phenomenon can be already observed in Tables \ref{tab:top-summary} and \ref{tab:command-summary} for commands \verb|ls| and \verb|grep| (e.g., \verb|alias ls='ls -G'| or \verb|alias grep='grep -i'|).
Overall, aliases defining default arguments make up a sizable amount of \num{502801} (\per{10.5}) of all alias definitions. 
Interestingly, a majority of these aliases, \num{278370} (\per{55.4}), deal with enabling color in commands (e.g., \verb|--color=auto|, \verb|--color=always|).
The top 3 aliases (or 6 depending on whether you would want to count basic synonyms separately) make up more than \per{95} of the \per{55.4}: \verb|dir|/\verb|vdir| (\per{21.7}), \verb|ls| (\per{26.5}), and \verb|grep|/\verb|fgrep|/\verb|egrep| (\per{49.3}).
Since aliases defining color as their default argument would disproportionally skew any relative measure from other data we observed, we exclude them when reporting summary statistics for the remainder of this section.

To a much lesser extent, we saw two other patterns emerging: (1) enabling interactivity in commands, and (2) enabling human-readable and verbose outputs.
We also saw safety and interactivity as part of a broader trend beyond just aliases for default arguments and report more results on this pattern in Section \ref{sec:safety}.
To quickly summarize the extent within default arguments: \num{64497} aliases define interactivity arguments (or \per{28.7} of overall default argument aliases without color options).
Default arguments that enable human-readable and verbose outputs were present in \num{35080} aliases (or \per{15.6} of overall default argument aliases without color options).
%Table \ref{tab:default-arguments-overview} provides an overview of the most common aliases defining default arguments (when removing color options).



\subsection{Autocorrect}

Another trend we observed in our exploration was that of aliases introducing manually defined autocorrect rules for misspelled commands (e.g., \verb|gti| instead of \verb|git|).
While it is easy for the human eye to determine instances of these typographical errors, it is not as straightforward to formalize all different cases.
We opt for a conservative measure (potentially underestimating the true extent of the phenomenon) in which autocorrecting aliases are defined as alias names with the same string length as their associated commands with a distance measure above an empirically determined threshold.
We surveyed and experimented with different distance measures~\cite{navarro:01} and decided on using the Damerau-Levenshtein measure~\cite{damerau:64}.
It is a robust algorithm that in addition to tracking the number of insertions, deletions, and substitutions between two strings, it also captures the transposition of two characters, a common occurrence in misspelled commands.  
%We use the Jaro similarity metric~\cite{jaro:89}, as it is a widespread distance measure that was introduced to capture typographical errors.
%Querying alias names with the same length as command names, but differ in their content (i.e., they do not define default arguments) returns \num{518500} (\per{10.86}) results.
We query alias names with the same length as command names, but differ in their content (i.e., they do not define default arguments), apply the Damerau-Levenshtein algorithm, and normalize the result.
Normalized distances are a float in the range of 0 and 1, where 0 entails there are no differences between the two strings and 1 means the distance is as far as it can be (corresponding to the length of the longer string in the original score).
We compute the results, order by normalized score, and empirically determine $\frac{1}{3}$ (roughly corresponding to \per{33} difference between autocorrecting alias name and command name) to be considered a conservative threshold to determine misspellings.
This results in \num{63875} (\per{1.3}) total aliases that serve as autocorrect rules.
Table \ref{tab:autocorrect} provides an overview of interesting autocorrecting aliases with different distance measures.

\begin{table}[]
	\caption{A curated selection from autocorrecting aliases highlighting a variety of use cases together with their normalized Damerau-Levenshtein distance}
		\label{tab:autocorrect}
	\begin{tabular}{@{}llll@{}}
		\toprule
		\textbf{Commands}                                                      & \textbf{Alias}                                                       & \textbf{Reason}      & \textbf{Score} \\ \midrule
		grep, sudo                                                             & grpe, sduo                                                           & Transposition        & 0.25           \\ \hline
		pluralise                                                              & pluralize                                                            & Localized Spelling   & 0.11           \\ \hline
		\begin{tabular}[c]{@{}l@{}}docker\_build, \\ mysql-server\end{tabular} & \begin{tabular}[c]{@{}l@{}}docker-build,\\ mysql.server\end{tabular} & Punctuation          & 0.08           \\ \hline
		jupyter                                                                & Jupyter                                                              & Case-sensitivity     & 0.14           \\ \hline
		gunzip                                                                 & ungzip                                                               & Mismatched casing & 0.33           \\ \bottomrule
	\end{tabular}
\end{table}



\input{complexity}

\subsection{Safety and Interactive Mode}
\label{sec:safety}

The \texttt{sudo} command allows the user to execute another command with superuser privileges.
In our dataset, we found \num{175806} commands (\per{6.75}) prefixed with \texttt{sudo}, used in \num{158779} alias definitions (\per{6.8}).
The top 10 commands used with \texttt{sudo} are given in \Cref{tab:sudo-commands}.

Remarkably, 9 out of the top 10 \texttt{sudo}-prefixed commands are related to package managers:
\texttt{zypper} (openSUSE); \texttt{pacman}, \texttt{abs} and \texttt{aur} (Arch Linux); \verb|apt-get| and \verb|$apt-pref| (Debian, Ubuntu); \texttt{yum} (RPM); \texttt{dnf} (Fedora); and \texttt{port} (macOS).
Even more remarkable: all of these commands are used with \texttt{sudo} the majority of the time they are used.
Some users even go so far as to redefine certain commands to always be executed with \texttt{sudo}, mostly system utilities like \texttt{reboot} or \texttt{shutdown}, but also (again) package managers.% like \texttt{pacman} and \texttt{apt}.
%However, on the whole this practice appears to be relatively rare (see \cref{tab:sudo-redefine}).

\begin{table}
    \caption{Top 10 commands used with \texttt{sudo}.}
    \label{tab:sudo-commands}
    \begin{tabular}{lrrl}
      \toprule
      Command & \# & \multicolumn{1}{c}{With \texttt{sudo}} & Description \\
      \midrule  
      \verb|zypper|    & \num{33753} & \num{31493} (\per{93.30})  & package manager\\
      \verb|pacman|    & \num{25322} & \num{17422} (\per{68.80})  & package manager \\
      \verb|apt-get|   & \num{16562} & \num{13946} (\per{84.20})  & package manager \\
      \verb|$apt_pref| & \num{10464} & \num{10464} (\per{100.00}) & package manager \\
      \verb|dnf|       & \num{12302} &  \num{6459} (\per{52.50})  & package manager \\
      \verb|yum|       & \num{11856} &  \num{6428} (\per{54.22})  & package manager \\
      \verb|port|      & \num{5574} &   \num{5416} (\per{97.17})  & package manager \\
      \verb|abs|       & \num{5097} &   \num{5094} (\per{99.94})  & package manager \\
      \verb|aur|       & \num{4525} &   \num{4489} (\per{99.20})  & package manager \\
      \verb|systemctl| & \num{7399} &   \num{3855} (\per{52.10})  & system utility \\
      \bottomrule
    \end{tabular}
\end{table}

On the other end of the safety spectrum, some users default to running the file system commands \texttt{rm}, \texttt{cp}, and \texttt{mv} in interactive mode, which prompts before performing potentially destructive actions.
Taking into account variations between systems and different ways to enable interactive mode, \per{36.83} of alias definitions invoking \texttt{rm} are redefinitions of \texttt{rm} enabling interactive mode by default.
Those numbers are even higher for \texttt{cp} and \texttt{mv}, where \per{65.97} and \per{74.49} of uses are redefinitions, respectively (see \Cref{tab:interactive}).

\begin{table}
    \caption{Commands commonly used in interactive mode}
    \label{tab:interactive}
    \begin{tabular}{lrrr}
        \toprule
        Command & \# & \multicolumn{1}{c}{With \texttt{-i}} & \multicolumn{1}{c}{Redefined With \texttt{-i}} \\
        \midrule
        rm & \num{20451} & \num{7878} (\per{38.52}) & \num{7533} (\per{36.83}) \\
        cp & \num{11573} & \num{7755} (\per{67.00}) & \num{7635} (\per{65.97}) \\
        mv & \num{9328} & \num{7068} (\per{75.77}) & \num{6948} (\per{74.49}) \\
        \bottomrule
    \end{tabular}
\end{table}


\subsection{Bookmarks}

We define a \emph{bookmark} to be an alias containing an argument that references some specific local or remote location, e.g. a file path, domain or IP address.
For instance, 
\begin{CVerbatim}
alias dl="cd ~/Downloads"
\end{CVerbatim}
and
\begin{CVerbatim}
alias starwars="telnet towel.blinkenlights.nl"
\end{CVerbatim}
are both bookmark aliases.

To find such bookmarks in our dataset, we searched for arguments that are locations, which we take to be any of the following:
\begin{itemize}
    \item An IPv4 address, matched by the liberal regular expression \verb|[0-9]+.[0-9]+.[0-9]+.[0-9]+|
    \item A string containing one of the known top-level domains\footnote{From \url{http://data.iana.org/TLD/tlds-alpha-by-domain.txt} (retrieved January 14, 2020)} preceded by a dot (\verb|.|) and followed by a slash (\verb|/|), colon (\verb|:|) or the end of the string.
    \item A string containing a forward slash (\verb|/|), indicating a path.
\end{itemize}
To avoid false positives, we sampled the top 300 search results according to the above criteria and determined some exclusion patterns.
For instance, \texttt{/dev/null} is not a location for our purposes.
Neither is \texttt{origin/master}, and thus 
\begin{CVerbatim}
alias gm="git merge origin/master"
\end{CVerbatim}
does not count as a bookmark.

By our definition, \num{557252} aliases (\per{11.68}) are bookmarks.
Of these, only \num{75169} are remote bookmarks (containing URLs or IP addresses).
Bookmarks are used predominantly for file system navigation, and the \texttt{cd} command is featured heavily in bookmark aliases.
Most other uses seem to be development related, like starting services such as web servers or databases with pre-defined locations, or outputting log files, as in 
\begin{CVerbatim}
alias onoz="cat /var/log/errors.log"
\end{CVerbatim}
or opening frequently edited files, as in the most common bookmark overall, which is for editing the shell configuration itself:
\begin{CVerbatim}
alias ohmyzsh="mate ~/.oh-my-zsh"
\end{CVerbatim}

In \cref{tab:use-cases} we have marked commands 




\section{Usability Smell Indicators}

Take a deeper dive and classify some of the more common commands and command groups along the axes defined in the previous section. These commands could be:
\begin{itemize}
	\item Unix System Tools (cd, ls, cp, mv, grep, rm, chmod, etc.)
	\item Git, Hg (Version control)
	\item Docker + docker compose, Kubernetes (Infrastructure)
	\item Package Managers (apt-get, zypper, etc.)
	\item Python, Ruby (Devops stuff? Runtime?)
\end{itemize}


\newcommand{\rot}[1]{\makebox[1em][l]{\rotatebox{45}{#1}}}

\newcommand{\yes}{$\CIRCLE$}
\newcommand{\no}{}
\newcommand{\some}{$\Circle$}
\newcommand{\many}{$\LEFTcircle$}

\begin{table*}
    \begin{tabular}{llrrllll}
        & & \# & \% & \rot{Default Arguments} & \rot{Autocorrect} & \rot{Compression} & \rot{Safety}  \\
        \midrule
        \multicolumn{2}{l}{Unix System Tools}           & \num{1234567} & \num{5.23} & \many & \some & \no &       \\
            & \texttt{cd}                               & \num{2323}    & \num{5.23} &       & \yes  &     & \yes  \\
            & \texttt{ls}                               & \num{2323}    & \num{5.23} & \yes  &       &     &       \\
            & \texttt{cp}                               & \num{2323}    & \num{5.23} & \yes  &       &     &       \\
            & \texttt{mv}                               & \num{2323}    & \num{5.23} & \yes  &       &     &       \\
            & \texttt{rm}                               & \num{2323}    & \num{5.23} & \yes  &       &     &       \\
            & \texttt{grep}                             & \num{2323}    & \num{5.23} & \yes  &       &     &       \\
            & \texttt{chmod}                            & \num{2323}    & \num{5.23} & \yes  &       &     &       \\
        \midrule
        \multicolumn{2}{l}{Version Control}             & \num{1234567} & \num{5.23} & \many & \some & \no &       \\
            & \texttt{git}                              & \num{2323}    & \num{5.23} &       & \yes  &     & \yes  \\
            & \texttt{hg}                               & \num{2323}    & \num{5.23} & \yes  &       &     &       \\
        \midrule
        \multicolumn{2}{l}{Container Infrastructure}    & \num{1234567} & \num{5.23} & \many & \some & \no &       \\
            & \texttt{docker}                           & \num{2323}    & \num{5.23} &       & \yes  &     & \yes  \\
            & \texttt{kubernetes}                       & \num{2323}    & \num{5.23} & \yes  &       &     &       \\
        \midrule
        \multicolumn{2}{l}{Package Managers}            & \num{1234567} & \num{5.23} & \many & \some & \no &       \\
            & \texttt{apt}                              & \num{2323}    & \num{5.23} &       & \yes  &     & \yes  \\
            & \texttt{zypper}                           & \num{2323}    & \num{5.23} & \yes  &       &     &       \\
            & \texttt{pacman}                           & \num{2323}    & \num{5.23} & \yes  &       &     &       \\
        \midrule
        \multicolumn{2}{l}{Developer Tools}             & \num{1234567} & \num{5.23} & \many & \some & \no &       \\
            & \texttt{python}                           & \num{2323}    & \num{5.23} &       & \yes  &     & \yes  \\
            & \texttt{ruby}                             & \num{2323}    & \num{5.23} & \yes  &       &     &       \\
            & \texttt{xcode}                            & \num{2323}    & \num{5.23} & \yes  &       &     &       \\
            & \texttt{mate}                             & \num{2323}    & \num{5.23} & \yes  &       &     &       \\
        \midrule
    \end{tabular}
\end{table*}

% for potential related work (including method), google 'usability smells'
% e.g. https://www.sciencedirect.com/science/article/pii/S1071581916301215
%      https://link.springer.com/chapter/10.1007/978-3-662-44811-3_13

\section{Implications}

Through our large-scale analysis 
%of unique files that contain over 2.2 million alias definitions, 
we gained insight into practices detailing how users customize their command-line interface.
Based on our observations, we outline discussion points that go beyond single customization practices and identify implications that can address shortcomings in command-line user experience.
Further, while our presented findings already give us an understanding of customization practices over many different kinds of commands, we view our collected dataset as a playground for fine-grained discovery that can benefit researchers, tool builders, and command-line users.

\paragraph{\bf Learning Repair Rules}

The complexity of commands and arguments can cause users to introduce errors when working in a command line interface.
Figuring out specifically how to fix these errors is often a convoluted process.
There is a popular open source project that attempts to navigate this issue.\footnote{\url{https://github.com/nvbn/thefuck}}
It uses a set of rules to suggest possible error corrections to the command.
While these rules are all hard-coded, we envision leveraging the global wisdom of customizations in our large scale dataset to learn rules that form the basis for different kinds of suggestions.
%An obvious choice from our analysis are autocorrecting aliases to learn repair rules.
\paragraph{\bf Discovering Workflows}
We can also see how our dataset would enable a world beyond only trying to fix immediate errors, but also provide usage hints that could introduce users to common parameters and workflows overall.
\TODO example for workflow
%\paragraph{\bf Object Protocols}
Similar to workflows, we have also seen the prevalence of object protocols~\cite{beckman:11}, which are implicit rules determining  the order in which commands have to be executed.
We can improve usability by enabling the discovery of these rules and exposing the dependency structure from our customization data.
For instance, if executing \verb|brew upgrade| results in a failure, we can suggest using \verb|brew update && brew upgrade| instead based on the customization patterns we found on chaining subcommands.

\paragraph{\bf Uncovering Conceptual Design Flaws}

Customization can also be an indicator for problems in the underlying conceptual design that manifest as usability frustrations that require adaption.
e want to briefly compare to the qualitative analysis on conceptual design flaws in \verb|git| by Perez and Jackson~\cite{perez:13}.
Particularly, we found the frustrations with committing and switching branches\footnote{``Just Let Me Commit!" and ``I Just Want to Switch Branches" in the paper~\cite{perez:13}} presented as in many customizations in our dataset.
\TODO list the particular connections from our data to the paper

We want to emphasize that we are not suggesting that data on large-scale customization practice can replace qualitative analysis, but rather that the corpus we provide insights for exploration and usability research.

\paragraph{\bf Contextual Defaults}

Choosing proper defaults in user interfaces is a pillar of user experience design~\cite{nielsen2005power}.
We see approximately 14\% (XXX) of our customizations are overriding default use in various ways.
%This practice is invasive. 
It means \emph{all} uses of this command are overwritten to adhere to the customization. 
%We can see different interpretations for this practice.
We see overriding defaults as an indication that the assumed default does not match the usage profile.
The variety of different defaults in the data indicate what we call contextual defaults, where context could be the expertise level of a command-line user or a certain persona (e.g., sysadmin, data scientist, software engineer).
%Different kinds of users require different default entry points from a command.

We could imagine providing a different set of aliases to different users (e.g., alias starter packs for different levels of expertise) generated from our data.

%We  a particular argument of a command (or set of arguments) repeatedly show up in customizations where that command is redefined as a new default,  
%it could be an indicator for the repeated argument structure to become a default.



\paragraph{\bf Interactive Shell vs Batch-Processing}

Context can be also seen as the environment a command is executed in. 
There is a difference in use of commands as part of scripting for batch-processing, compared to interactive use on the command-line.
Our findings highlight the tension between commands used in scripting for automation, and as part of an interactive medium to interface with the operating system.
This particularly comes to light when highlighting our findings on customizations that redefine commands with defaults for interactive safety:
While it makes no sense to use the \verb|--interactive| or \verb|-i| argument, our findings show that it is a highly desired customization practice for interactive command-line use.


\section{Threats to Validity}

We review potential limitations of our study as threats to validity.
First, our sample might not be representative.
Our dataset only includes aliases by people who publicly shared their dotfiles, we only collected from GitHub, and our sample does not include forks.
Nevertheless, our dataset is very exhaustive, as we were able to sample \per{94.09} of the estimated population of Shell files containing aliases on GitHub.
And while mining GitHub can be fraught with perils \cite{kalliamvakou:14}, we specifically sought out personal repositories, side-stepping many of the typical issues with mining GitHub for software projects.

Second, our parser might not be sophisticated enough to recognize complex real-world aliases or cope with minute platform differences.
To mitigate this threat, we ran multiple sanity checks and tested the parser on some hairy examples from the dataset.
We did not detect any significant mis-parses and think that we have covered the majority of relevant cases.
The raw unparsed database is available in our replication package.

Third, aliases might not reflect intent as much as we assume.
En-masse copy-pasting of aliases by users, without them knowing exactly what they are copying, is certainly a realistic scenario.
System distributions and configuration frameworks like \emph{ohmyzsh} ship with numerous aliases by default or as part of easily enabled plugins.
Users might not even be aware of the aliases they have on their system.
We mitigate this concern by removing all duplicate files from our dataset that would indicate sheer copy/pasting.
%We also particularly exclude alias definitions that come bundled with operating systems and particular shells (e.g., zsh).
%This is mitigated by the fact that all of the aliases we collected were publicly shared by users.
%It stands to reason that even if a user is not aware of all the details of their system configuration, they confirm their attachment to this configuration and its aliases by publicly sharing them---even if only for the purposes of synchronizing them across the users' own machines.

Fourth, we might not actually be able to see the true user intent, if it exists, as quantitative measures might hide a long tail of minor variations and individual user preference.
Conclusions about common aliases or selected subsets might not be generalizable.
To mitigate these summarizing effects, we established customization practices as a vehicle to take a deeper dive into the details of certain alias usage.
Since we sampled almost the whole available population, we are confident in the strength of our data and the conclusions we can draw from particular instances.
Our replication package includes our whole toolchain and all alias data in a relational format ready for further analysis.


\section{Related Work}

Related research in the broader context of our work has been conducted on the use of use of command-line interfaces and how to improve it, and the shell as a programming language for both scripting and interactive use. 

The earliest related study we found was by Greenberg~\cite{greenberg:88a}, which collected four months of continuous real-life use of the Unix \verb|csh| shell from 168 users. 
The data was used in a follow up study to analyze the use of interactive systems by examining frequency of command invocations for different groups of users~\cite{greenberg:88b}.
In later work, Davison and Hirsh use probabilistic action modeling to predict user action sequences based on the same dataset~\cite{davison:98}.
Korvemaker and Greiner similarly predict future action sequences in command lines, but condition on actions of the particular user group with the goal of enabling adaptive user interfaces~\cite{korvemaker:00}.
Other work in the context of adaptive user interfaces by Jacobs and Blockeel uses association rule learning on the shell logs to produce scripts to automate common task sequences~\cite{jacobs:01}.
Khosmood et al. uses the same corpus and two additional, more recent, corpora to learn a model that can identify user profiles based on their command line behavior~\cite{khosmood:14}.
Bespoke~\cite{bespoke:19} is a system that synthesizes specialized graphical user interfaces (GUIs) based on command usage.
Our work can be viewed as an input to this system that passes common shell workflows in aliases to be generated as GUIs.

There has been other work on enhancing user experience in command line interfaces.
NoFAQ~\cite{dantoni:17} provides repair suggestions for failed shell invocations based on a model learned from a curated set of fix patterns.
NL2Bash~\cite{lin:18} implements a system that translates natural language phrases in English to shell commands.
Recent work by Greenberg has been looking into understanding the POSIX shell as a programming language~\cite{greenberg:17}.
More specifically, understanding word expansion in the shell to support interactivity~\cite{greenberg:18a} and concurrency~\cite{greenberg:18b}.


\section{Conclusion}

We report on a large scale exploratory study on how command line users customize user experience by defining shell aliases.
Through inductive coding, six use cases emerged from our dataset that provide insight on the characteristics of alias use.
Based on our results, we discuss and formulate a set of implications for command line tool developers, and the shell as an interactive environment for experts.

Aliases often redefine commands with default arguments, which is a potential indicator for usability problems in these tools.
However, we have to also be aware that defaults can be highly contextual depending on user profiles (e.g., expertise level) and environment (e.g., scripting vs. interactive use).
We also see our dataset and results as a rich source for learning norms with respect to repair rules, data flows, and descriptive names for complex command structures.
We provide a comprehensive replication package and see potential for future work based on our dataset and analyses.

\bibliographystyle{spbasic}
\bibliography{references}

\end{document}
