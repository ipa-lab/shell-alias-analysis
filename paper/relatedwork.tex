\section{Related Work}

Related research in the broader context of our work has been conducted on the use of command-line interfaces and how to improve it, and the shell as a programming language for both scripting and interactive use. 

The earliest related study we found was by \cite{greenberg:88a}, which collected four months of continuous real-life use of the Unix \verb|csh| shell from 168 users. 
The data was used in a follow up study to analyze the use of interactive systems by examining frequency of command invocations for different groups of users~\citep{greenberg:88b}.
In later work, \cite{davison:98} use probabilistic action modeling to predict user action sequences based on the same dataset.
\cite{korvemaker:00} similarly predict future action sequences in command lines, but condition on actions of the particular user group with the goal of enabling adaptive user interfaces.
Other work in the context of adaptive user interfaces by \cite{jacobs:01} uses association rule learning on the shell logs to produce scripts to automate common task sequences.
\cite{khosmood:14} use the same corpus and two additional, more recent, corpora to learn a model that can identify user profiles based on their command line behavior.
Bespoke~\citep{bespoke:19} is a system that synthesizes specialized graphical user interfaces (GUIs) based on command usage.
Our work can be viewed as an input to this system that passes common shell workflows in aliases to be generated as GUIs.

There has been other work on enhancing user experience in command line interfaces.
NoFAQ~\citep{dantoni:17} provides repair suggestions for failed shell invocations based on a model learned from a curated set of fix patterns.
NL2Bash~\citep{lin:18} implements a system that translates natural language phrases in English to shell commands.
Recent work by \cite{greenberg:17} has been looking into understanding the POSIX shell as a programming language.
More specifically, understanding word expansion in the shell to support interactivity~\citep{greenberg:18a} and concurrency~\citep{greenberg:18b}.
