Shell aliases allow command line users to customize their environment by defining command substitutions.
When processing the command line, the shell replaces each alias name with its string value, which is in turn executed.
%For instance, the common alias \verb|ll| may be replaced with \verb|ls -l|.
The definitions can take the form of any arbitrarily complex command or chain of commands.
To characterize the patterns and complexities of shell customization through aliases, we conducted an exploratory empirical study on over 2.2 million alias definitions found on GitHub.
We found that aliases enable a number of complex customization scenarios: enforcing default arguments, enabling safety and interaction, storing bookmarks to local and remote locations, and providing autocorrect for misspelled commands.
When analyzing aliases with respect to compression, they are not only used to abbreviate and simplify longer and complicated command and argument structures, but also to lend them more descriptive (and longer) names.
We conjecture that shell aliases can point to particular usability issues within command line programs, and that a deeper understanding of shell customization practices can support tool developers in designing better user experiences.