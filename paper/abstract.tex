Shell aliases allow command line users to customize their environment by defining command substitutions.
When processing the command line, the shell replaces each alias name with its string value, which is in turn executed.
%For instance, the common alias \verb|ll| may be replaced with \verb|ls -l|.
The definitions can take the form of any arbitrarily complex command or chain of commands.
To characterize the patterns and complexities of shell customization through aliases, we conducted an exploratory empirical study on over 2.2 million alias definitions found on GitHub.
Using inductive coding methods, we found that there are three types of aliases and that each enables a number of complex customization scenarios: 
\textsc{Shortcuts} (for \emph{nicknaming commands}, \emph{abbreviating subcommands}, and \emph{bookmarking locations}),
\textsc{Modifications} (for \emph{substituting commands}, \emph{overriding defaults}, \emph{colorizing output}, and \emph{elevating privilege}),
and \textsc{Scripts} (for \emph{transforming data} and \emph{chaining subcommands}).
\TODO We conjecture that shell aliases can point to particular usability issues within command line programs, and that a deeper understanding of shell customization practices can support tool developers in designing better user experiences.