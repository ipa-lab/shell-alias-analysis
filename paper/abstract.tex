The interactive command-line, also known as the shell, is a prominent mechanism used extensively by a wide range of software professionals (engineers, system administrators, data scientists, etc.). Shell customizations can therefore provide insight into the tasks they repeatedly perform, how well the standard environment supports those tasks, and ways in which the environment could be productively extended or modified.
To characterize the patterns and complexities of command-line customization, we conducted an exploratory empirical study on over 2.2 million shell alias definitions found on GitHub.
Shell aliases allow command line users to customize their environment by defining arbitrarily complex command substitutions.
Using inductive coding methods, we found three types of aliases that each enable a number of customization practices: 
\textsc{Shortcuts} (for \emph{nicknaming commands}, \emph{abbreviating subcommands}, and \emph{bookmarking locations}),
\textsc{Modifications} (for \emph{substituting commands}, \emph{overriding defaults}, \emph{colorizing output}, and \emph{elevating privilege}),
and \textsc{Scripts} (for \emph{transforming data} and \emph{chaining subcommands}).
We conjecture that identifying common customization practices can point to particular usability issues within command-line programs, and that a deeper understanding of these practices can support researchers and tool developers in designing better user experiences.