The interactive command-line is a prominent mechanism that a wide range of software professionals (engineers, sysadmins, data scientists, etc.) use extensively during development and other systems activities. Shell command customizations can therefore provide insight into the tasks that they repeatedly perform, how well the standard environment supports those tasks, and ways in which this standard environment could be productively extended or modified.
To characterize the patterns and complexities of command-line customization, we conducted an exploratory empirical study on over 2.2 million alias definitions found on GitHub.
Shell aliases allow command line users to customize their environment by defining (arbitrarily complex) command substitutions.
Using inductive coding methods, we found that there are three types of aliases and that each enables a number of complex customization scenarios: 
\textsc{Shortcuts} (for \emph{nicknaming commands}, \emph{abbreviating subcommands}, and \emph{bookmarking locations}),
\textsc{Modifications} (for \emph{substituting commands}, \emph{overriding defaults}, \emph{colorizing output}, and \emph{elevating privilege}),
and \textsc{Scripts} (for \emph{transforming data} and \emph{chaining subcommands}).
We conjecture that identifying common customization practices can point to particular usability issues within command-line programs, and that a deeper understanding of these practices can support researchers and tool developers in designing better user experiences.