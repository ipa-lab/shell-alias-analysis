\section{Background}

A shell is a command interpreter allowing the user to interact with an underlying system.
The concept of the operating system shell as an independent process executing outside the kernel originated in Multics and was further developed into the original Unix shell \texttt{sh} and its various descendants.
%  (\TODO references)
The POSIX family of standards defines a Shell Command Language~\citep{posix_standard, greenberg:17}, whose standard implementation is still the \texttt{sh} utility, but there exist a wide variety of popular POSIX-compliant shells like \texttt{bash} or \texttt{zsh}.
These implementations are free to extend the functionality of the shell, but all share a common subset of core commands and programming language constructs.
In this paper, we focus on the built-in \texttt{alias} command, available on all POSIX shells.

\subsection{Usage and Syntax}

The \texttt{alias} command allows the user to create \emph{alias definitions}, defining command substitutions.
When the shell processes the command line, it replaces known alias names with their defined string values.
For example, 
\begin{CVerbatim}
alias ll='ls -l'
\end{CVerbatim}
defines the \emph{alias name} \texttt{ll}, that is replaced by the \emph{alias value} \texttt{ls -l}.
In this case, \cmd{ls} is the standard command for listing directory contents, with the argument \texttt{-l} specifying a long-form output format.
So the alias \texttt{ll} (present in many system configurations) is used to specify a default argument to a commonly used command under a different name.

Alias values can be arbitrarily complex strings and can substitute not only simple commands and arguments, but whole chains of commands. 
The definition
\begin{CVerbatim}
alias ducks="du -cksh * | sort -hr | head -n 15"
\end{CVerbatim}
defines the new command \cmd{ducks} by chaining together three different command-line tools in order to return the 15 largest files in the current directory.

%alias ip=ifconfig | grep "inet " | grep -v 127.0.0.1 | cut -d' ' -f2
%which chains together several commands to find out the IP addresses of the system.

%alias unwip="git log -n 1 | grep -q -c wip && git reset HEAD~1"
%which chains together three commands to rewind the most recent commit in a git repository, if its commit message contains the word "wip".

In general, an alias definition takes the form
\begin{CVerbatim}
alias name=value
\end{CVerbatim}
where \verb|value| can optionally be enclosed in single (\verb|'|) or double (\verb|"|) quotes and \verb|name| can be any identifier that is a valid command name.\footnote{Some shells allow for an alternative alias syntax without the equals sign between \texttt{name} and \texttt{value}. In this paper we only look at POSIX-compliant alias definitions.}
% TODO: technically not quite correct, see FSE review 2
In particular, the alias name can be an existing command, so a re-definition like
\begin{CVerbatim}
alias grep='grep --color=always'
\end{CVerbatim}
is possible.

In the remainder of this paper, we will use the more compact notation
\begin{center}\alias{a}{b}\end{center} to indicate an alias that replaces the name \texttt{a} with the value \texttt{b}.

\subsection{Dotfiles}

Aliases can be entered directly on the command line, in which case they are valid until the shell session ends.
To make an alias definition permanent, it is common practice to enter it into a file that is read and executed by the shell on startup.
The names of these configuration files differ by shell, but common ones are \verb|.bashrc|, \verb|.zshrc| or \verb|.profile|.
Often, aliases are also stored in other files referred to by these startup scripts.

These kinds of files --- text-based configuration files that store system or application settings --- are also known as \emph{dotfiles}, because their filenames usually start with a dot (\verb|.|) so that they are hidden by default on most Unix-based systems.
In recent years, people have started sharing their dotfiles on platforms like GitHub.\footnote{\url{https://dotfiles.github.io}}
This has the advantage of being able to sync one's configurations across different machines, and also enables exchange and discovery of configurations between users.
