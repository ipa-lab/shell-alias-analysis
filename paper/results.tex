\section{Analysis}

\begin{tabular}{lrr}
\toprule
   Alias Name &           \# &          \% \\
\midrule
    \verb|ls| &  \num{83782} &   \num{3.8} \\
    \verb|ll| &  \num{62465} &  \num{2.83} \\
  \verb|grep| &  \num{44479} &  \num{2.02} \\
    \verb|la| &  \num{43760} &  \num{1.99} \\
     \verb|l| &  \num{39539} &  \num{1.79} \\
\bottomrule
\end{tabular}
\hspace{0.3cm}
\begin{tabular}{lrr}
    \toprule
           Command &            \# &           \% \\
    \midrule
        \verb|git| &  \num{327786} &  \num{12.93} \\
         \verb|ls| &  \num{260156} &  \num{10.27} \\
         \verb|cd| &  \num{166632} &   \num{6.58} \\
       \verb|grep| &   \num{89598} &   \num{3.54} \\
        \verb|vim| &   \num{46545} &   \num{1.84} \\
    \bottomrule
\end{tabular}
\hspace{0.3cm}
\begin{tabular}{lrr}
    \toprule
                Argument &            \# &          \% \\
    \midrule
     \verb|--color=auto| &  \num{153931} &  \num{4.24} \\
               \verb|-i| &   \num{70640} &  \num{1.95} \\
               \verb|-a| &   \num{42910} &  \num{1.18} \\
               \verb|-l| &   \num{39519} &  \num{1.09} \\
               \verb|-v| &   \num{35295} &  \num{0.97} \\
    \bottomrule
\end{tabular}



\Cref{tab:top-summary} shows the most common alias names, commands, and arguments appearing in alias definitions.
The most common alias name we found is \texttt{ls}, appearing a total number of \num{83782} times, which is \per{3.8} of all alias definitions.
Note that this is \texttt{ls} as an \emph{alias name}, a redefinition of the \texttt{ls} \emph{command}, which appears \num{260156} times (\per{10.27}).
This is a bit less often than \texttt{git}, the most common command, which appears in \num{327786} aliases (\per{12.93}).
The most common argument, across all commands, is \texttt{--color=auto}, appearing \num{153931} times (\per{4.24})

\input{tables/command-summary.tex}

\newcommand{\rot}[1]{\makebox[1em][l]{\rotatebox{45}{#1}}}

\newcommand{\full}{$\CIRCLE$}
\newcommand{\half}{$\LEFTcircle$}
\newcommand{\empt}{$\Circle$}

\newcommand{\hist}[1]{\includegraphics[height=1em, trim=1em 1em 1em 1em, clip]{compression/#1.pdf}}

\newcommand*{\pie}[1]{\begin{tikzpicture}[scale=0.15]%
    \draw (0,0) circle (1);
    \fill[fill opacity=1,fill=black] (0,0) -- (90:1) arc (90:90-#1*3.6:1) -- cycle;
    \end{tikzpicture}}    

\begin{table*}
    \caption{Common commands broken down by alias use cases. We display the percentage of a particular command belonging to a category with a pie chart symbol, if it is more than \per{1}. The distribution of compressions are log-log histograms. The red line marks compression ratio 1. Consequently, distributions to the left of the vertical line are actually \emph{expansions}.}
    \label{tab:use-cases}
    \begin{tabular}{llrlllllccc}
        & & \# & &\rot{Default Arguments} & \rot{Autocorrect} & \rot{Chaining} & \rot{Safety} & \rot{Bookmarks} & & Compression \\
        \midrule
        \multicolumn{2}{l}{Version Control} \\
            & \texttt{git}                                  & \num{629593} & & & & \pie{5.84} & &             & & \hist{git} \\
            & \texttt{hg}                                   & \num{29363} &  & & &	\pie{4.66}  & &             & & \hist{hg} \\
        \midrule
        \multicolumn{2}{l}{System Tools} \\
            & \texttt{ls}                                   & \num{384186} & & \pie{27.11} & & \pie{2.29} &             &             & & \hist{ls} \\
            & \texttt{cd}                                   & \num{229522} & & & & \pie{4.79} &             & \pie{63.37} & & \hist{cd} \\
            & \texttt{grep}*                                & \num{223629} &  & \pie{63.02} & & \pie{24.29} &       & \pie{1.51} & & \hist{grep} \\
            & \texttt{echo}                                 & \num{53934} &  & \pie{1.14} & & \pie{31.27} &             &  \pie{7.54} & & \hist{echo} \\
            & \texttt{xargs}                                & \num{44927} &  & & & \pie{35.27} &             &             & & \hist{xargs} \\
            & \texttt{ssh}                                  & \num{36574} &  & \pie{4.54} & & \pie{3.46} &             & \pie{64.39} & & \hist{ssh} \\
            & \texttt{rm}                                   & \num{44209} &  & \pie{48.29} & & \pie{13.02} & \pie{56.53} & \pie{22.68} & & \hist{rm} \\
            & \texttt{dir}                                  & \num{31069} &  & \pie{99.55} & & &             &             & & \hist{dir} \\
            & \texttt{cp}                                   & \num{27472} &  & \pie{76.35} & & \pie{4.72} & \pie{70.61} & \pie{12.62} & & \hist{cp} \\
            & \texttt{mv}                                   & \num{22689} &  & \pie{83.03} & & \pie{3.12} & \pie{79.21} & \pie{5.56}  & & \hist{mv} \\
            & \texttt{sort}                                 & \num{22391} &  & & & \pie{87.04} &             &             & & \hist{sort} \\
            & \texttt{head}                                 & \num{17530} &  & & & \pie{78.32} &             & \pie{1.04}  & & \hist{head} \\
            & \texttt{cat}                                  & \num{17425} &  & & & \pie{15.16} & \pie{1.81}  & \pie{42.48} & & \hist{cat} \\
        \midrule
        \multicolumn{2}{l}{Package Managers} \\
            & \texttt{zypper}                           & \num{66295} & & & & & \pie{93.36} &           & & \hist{zypper} \\
            & \texttt{pacman}                           & \num{46821} & & \pie{2.81} & & \pie{1.22} & \pie{69.21} &           & & \hist{pacman} \\
            & \texttt{mvn}                              & \num{36683} & & & & \pie{1.09} &             &           & & \hist{mvn} \\
            & \texttt{yaourt}                           & \num{34577} & & & & &             &           & & \hist{yaourt} \\
            & \texttt{apt}*                         & \num{74991} & &  \pie{9.8}&   & \pie{10.16} &  \pie{45.0}           &           & & \hist{apt} \\
            & \texttt{brew}                             & \num{27060} & & & & \pie{39.49} &             &           & & \hist{brew} \\
            & \texttt{port}                                 & \num{11228} &  & & &\pie{3.79}  & \pie{96.75} &             & & \hist{port} \\
        \midrule
        \multicolumn{2}{l}{Text Editors}  \\
            & \texttt{mate}                             & \num{61832} & & & & &             & \pie{95.77} & & \hist{mate} \\
            & \texttt{vim}                              & \num{53378} &     & \pie{3.28} & & \pie{3.28} & \pie{5.1}   & \pie{44.02} & & \hist{vim} \\
            & \texttt{nvim}                             & \num{19006} &     & & & \pie{1.19} & \pie{1.72}  & \pie{17.38} & & \hist{nvim} \\
            & \texttt{emacs}                            & \num{10299} & & \pie{18.44} & \pie{10.75} & \pie{1.16} & \pie{2.19}  & \pie{10.83} & & \hist{emacs} \\
        \midrule
        \multicolumn{2}{l}{Developer Tools} \\
            & \texttt{wp}                               & \num{66434} & & & & &             &             & & \hist{wp} \\
            & \texttt{zeus}                             & \num{52570} & & & & \pie{12.09} &             & \pie{23.91} & & \hist{zeus} \\
            & \texttt{php}                              & \num{44486} & & & & \pie{7.06}&             & \pie{6.9}   & & \hist{php} \\
        \midrule
        \multicolumn{2}{l}{Infrastructure} \\
            & \texttt{docker}*  & \num{73706} & & & & \pie{3.86} & \pie{2.63} & \pie{7.6} & & \hist{docker} \\
            & \texttt{kubectl}*                                      & \num{39781} & & & & & & & & \hist{kubectl} \\
            & \texttt{vagrant}                                                  & \num{8953} &  & & & \pie{11.17} & & & & \hist{vagrant} \\
        \midrule
        \multicolumn{2}{l}{Other} \\
            & \texttt{ffmpeg}                             & \num{606} & & \pie{14.69} & & \pie{8.75} &            & \pie{30.2} & & \hist{ffmpeg} \\
            & \texttt{beep}                               & \num{85} &  & \pie{4.71} & & \pie{50.59} & \pie{4.71} &            & & \hist{beep} \\
    \end{tabular}
\end{table*}
 % TODO: update

Looking at each part of an alias definition in isolation can only get us so far, as arguments only gain meaning in conjunction with commands and alias names can be identical between users, referring to the same command/argument combination, or indeed can overlap, meaning the same alias name is used differently by different users.
\Cref{tab:command-summary} gives a more informative view for the top two commands, \texttt{git} and \texttt{ls}, showing us the top arguments given with each and the most common alias names by which the command/argument combinations are referred to.
Here we can already identify some of the different customization practices involving aliases.
Looking at \texttt{ls}, we find that aliases are used
to redefine the command with a default argument (\verb|alias ls="ls --color=auto"|);
to shorten a common invocation (\verb|alias ll="ls -alF"|);
and to correct a spelling mistake (\verb|alias sl=ls|).
We also notice that in the case of \texttt{git}, most aliases are used for shortening \texttt{git} subcommand invocations (e.g. \verb|alias gd="git diff"|).

%TODO describe coding for use cases / config practices


% TODO: update numbers below
Aliases are used differently for different commands and indeed different command categories.
We looked at the top 200 aliased commands and identified roughly six command categories: system tools, like \texttt{ls} or \texttt{grep}, but also \texttt{ssh} (making up \per{46.12} of the top aliased commands); version control, like \texttt{git} (\per{15.61}); development tools, like compilers or databases (\per{14.09}); package managers, both system-wide and for specific programming environments (\per{12.05}); text editors (\per{4.27}); infrastructure tools, like \texttt{docker}, \texttt{kubernetes} or \texttt{vagrant} (\per{3.02}). The rest (\per{4.83}) were deemed unclassifiable.
\Cref{tab:use-cases} presents a selection of commands from each of these categories and gives a quantitative overview about the kinds of aliases they are commonly used in.
For each command, it shows how many uses of that command involve one of the different use cases.
In addition to the three use cases already mentioned (default arguments, compression and autocorrect), we identified another three (command chaining, safety and bookmarks) and will now discuss each of these six use cases in more detail.


%The single most common alias definition in our dataset is
\begin{verbatim}
alias '${1+"$@"}'="$@"
\end{verbatim}
appearing \numprint{78265} times, which is \numprint[\%]{1.64} of all alias definitions.
This curious jumble of symbols derives from a portability hack for \texttt{zsh}.\footnote{\url{https://unix.stackexchange.com/a/191836}}

More familiar results can be seen in \Cref{tab:top-summary}, which shows the most common alias names, commands, and arguments appearing in alias definitions.
The most common alias name we found is \texttt{ls}, appearing a total number of \numprint{124515} times, which is \numprint[\%]{2.61} of all alias definitions.
Note that this is \texttt{ls} as an \emph{alias name}, a redefinition of the \texttt{ls} \emph{command}, which appears \numprint{378097} times (\numprint[\%]{7.12}).
This is about half as often as \texttt{git}, the most common command, which appears in \numprint{626185} aliases (\numprint[\%]{11.79}).
The most common argument, across all commands, is \texttt{--color=auto}, appearing \numprint{268171} times (\numprint[\%]{3.6})

\begin{table*}
    \caption{Top alias names, commands and arguments}
    \label{tab:top-summary}
    \begin{tabular}{lrr}
        \toprule
                 Alias Name &                 \# &               \% \\
        \midrule
                  \verb|ls| &  \numprint{124515} &  \numprint{2.61} \\
                  \verb|ll| &   \numprint{89254} &  \numprint{1.87} \\
         \verb|'${1+"$@"}'| &   \numprint{78265} &  \numprint{1.64} \\
                \verb|grep| &   \numprint{63392} &  \numprint{1.33} \\
                  \verb|la| &   \numprint{62039} &  \numprint{1.30} \\
        \bottomrule
    \end{tabular}
    \hspace{0.1cm}
    \begin{tabular}{lrr}
        \toprule
               Command &                 \# &                \% \\
        \midrule
            \verb|git| &  \numprint{626185} &  \numprint{11.79} \\
             \verb|ls| &  \numprint{378097} &   \numprint{7.12} \\
             \verb|cd| &  \numprint{229522} &   \numprint{4.32} \\
           \verb|grep| &  \numprint{128045} &   \numprint{2.41} \\
           \verb|"$@"| &   \numprint{78276} &   \numprint{1.47} \\
        \bottomrule
    \end{tabular}
    \hspace{0.1cm}
    \begin{tabular}{lrr}
        \toprule
                    Argument &                 \# &               \% \\
        \midrule
         \verb|--color=auto| &  \numprint{268171} &  \numprint{3.60} \\
                   \verb|-i| &  \numprint{109094} &  \numprint{1.47} \\
                   \verb|-v| &   \numprint{70424} &  \numprint{0.95} \\
                   \verb|-l| &   \numprint{63302} &  \numprint{0.85} \\
                   \verb|-a| &   \numprint{62023} &  \numprint{0.83} \\
        \bottomrule
    \end{tabular}        
\end{table*}

Looking at each part of an alias definition in isolation can only get us so far, as arguments only gain meaning in conjunction with commands and alias names potentially vary widely.
\Cref{tab:command-summary} gives a more informative view, centered around some common commands, showing us the top arguments given with each command and the most common alias names by which the command/argument combinations are referred to.
Here we can see \TODO

\begin{table*}
    \caption{Common commands and their top arguments and aliases}
    \label{tab:command-summary}
    \begin{tabular}{lrll}
        \toprule
             Command &                \% &                 Arguments &                                                                                 Aliases (\%) \\
        \midrule
          \verb|git| &   \numprint{3.75} &             \verb|status| &                                  \verb|gs| (\numprint{51.57}), \verb|gst| (\numprint{24.17}) \\
                     &   \numprint{2.42} &           \verb|checkout| & \verb|gco| (\numprint{51.17}), \verb|gc| (\numprint{10.04}), \verb|letcat| (\numprint{9.18}) \\
                     &   \numprint{2.33} &               \verb|push| &    \verb|gp| (\numprint{49.12}), \verb|rulz| (\numprint{9.53}), \verb|gps| (\numprint{7.15}) \\
                     &   \numprint{2.31} &                   \verb|| &                                                                  \verb|g| (\numprint{78.87}) \\
                     &   \numprint{2.18} &               \verb|pull| &    \verb|gl| (\numprint{25.32}), \verb|gpl| (\numprint{13.29}), \verb|gp| (\numprint{11.53}) \\
                     &   \numprint{2.11} &               \verb|diff| &                                                                 \verb|gd| (\numprint{83.21}) \\
                     &   \numprint{2.05} &                \verb|add| &                              \verb|ga| (\numprint{73.01}), \verb|chicken| (\numprint{10.85}) \\
                     &   \numprint{2.01} &             \verb|branch| &                                   \verb|gb| (\numprint{74.88}), \verb|gbr| (\numprint{8.26}) \\
                     &   \numprint{1.59} &          \verb|commit -m| & \verb|gcm| (\numprint{21.67}), \verb|gcmsg| (\numprint{19.59}), \verb|gc| (\numprint{18.18}) \\
                     &   \numprint{1.20} &             \verb|commit| &                                                                 \verb|gc| (\numprint{65.51}) \\
        \midrule
           \verb|ls| &  \numprint{13.15} &       \verb|--color=auto| &                                                                 \verb|ls| (\numprint{99.09}) \\
                     &   \numprint{9.12} &                 \verb|-A| &                                                                 \verb|la| (\numprint{98.27}) \\
                     &   \numprint{8.57} &                \verb|-CF| &                                                                  \verb|l| (\numprint{98.94}) \\
                     &   \numprint{6.69} &                 \verb|-l| &                                                                 \verb|ll| (\numprint{82.64}) \\
                     &   \numprint{6.63} &               \verb|-alF| &                                                                 \verb|ll| (\numprint{97.89}) \\
                     &   \numprint{3.65} &                   \verb|| &      \verb|l| (\numprint{21.38}), \verb|sl| (\numprint{16.99}), \verb|iz| (\numprint{12.24}) \\
                     &   \numprint{3.06} &                 \verb|-G| &                                                                 \verb|ls| (\numprint{97.46}) \\
                     &   \numprint{2.43} &                \verb|-la| &      \verb|ll| (\numprint{32.80}), \verb|la| (\numprint{22.61}), \verb|l| (\numprint{12.71}) \\
                     &   \numprint{2.01} &                 \verb|-a| &                                                                 \verb|la| (\numprint{74.80}) \\
                     &   \numprint{1.77} &               \verb|-lah| &     \verb|l| (\numprint{32.64}), \verb|lsa| (\numprint{31.61}), \verb|ll| (\numprint{19.01}) \\
        \midrule
         \verb|grep| &  \numprint{36.12} &       \verb|--color=auto| &                                                               \verb|grep| (\numprint{99.29}) \\
                     &  \numprint{12.50} &                   \verb|| &   \verb|G| (\numprint{22.27}), \verb|hgrep| (\numprint{10.98}), \verb|rrg| (\numprint{7.22}) \\
                     &   \numprint{4.42} &            \verb|--color| &                                                               \verb|grep| (\numprint{97.00}) \\
                     &   \numprint{1.68} &                 \verb|-i| &        \verb|grep| (\numprint{6.88}), \verb|G| (\numprint{5.86}), \verb|g| (\numprint{5.86}) \\
                     &   \numprint{1.52} & \verb|--color=never '^d'| &                                                                \verb|lsd| (\numprint{98.45}) \\
        \bottomrule
    \end{tabular}
\end{table*}

One drawback of this presentation is that there are obviously many different ways of how an arguments to a command can be combined, since for most programs, the order of arguments does not matter and often there are multiple ways to specify the same argument.
\TODO


\subsection{Default Arguments}

One common use case of using aliases was to define commands with default arguments.
More specifically, we refer to an alias as defining default arguments when the alias name is the same as the command name, and in the alias definition, the command specifies at least one further argument.
This phenomenon can be already observed in Tables \ref{tab:top-summary} and \ref{tab:command-summary} for commands \verb|ls| and \verb|grep| (e.g., \verb|alias ls='ls -G'| or \verb|alias grep='grep -i'|).
Overall, aliases defining default arguments make up a sizable amount of \num{502801} (\per{10.5}) of all alias definitions. 
Interestingly, a majority of these aliases, \num{278370} (\per{55.4}), deal with enabling color in commands (e.g., \verb|--color=auto|, \verb|--color=always|).
The top 3 aliases (or 6 depending on whether you would want to count basic synonyms separately) make up more than \per{95} of the \per{55.4}: \verb|dir|/\verb|vdir| (\per{21.7}), \verb|ls| (\per{26.5}), and \verb|grep|/\verb|fgrep|/\verb|egrep| (\per{49.3}).
Since aliases defining color as their default argument would disproportionally skew any relative measure from other data we observed, we exclude them when reporting summary statistics for the remainder of this section.

To a much lesser extent, we saw two other patterns emerging: (1) enabling interactivity in commands, and (2) enabling human-readable and verbose outputs.
We also saw safety and interactivity as part of a broader trend beyond just aliases for default arguments and report more results on this pattern in Section \ref{sec:safety}.
To quickly summarize the extent within default arguments: \num{64497} aliases define interactivity arguments (or \per{28.7} of overall default argument aliases without color options).
Default arguments that enable human-readable and verbose outputs were present in \num{35080} aliases (or \per{15.6} of overall default argument aliases without color options).
%Table \ref{tab:default-arguments-overview} provides an overview of the most common aliases defining default arguments (when removing color options).



\subsection{Autocorrect}

Another trend we observed in our exploration was that of aliases introducing manually defined autocorrect rules for misspelled commands (e.g., \verb|gti| instead of \verb|git|).
While it is easy for the human eye to determine instances of these typographical errors, it is not as straightforward to formalize all different cases.
We opt for a conservative measure (potentially underestimating the true extent of the phenomenon) in which autocorrecting aliases are defined as alias names with the same string length as their associated commands with a distance measure above an empirically determined threshold.
We surveyed and experimented with different distance measures~\cite{navarro:01} and decided on using the Damerau-Levenshtein measure~\cite{damerau:64}.
It is a robust algorithm that in addition to tracking the number of insertions, deletions, and substitutions between two strings, it also captures the transposition of two characters, a common occurrence in misspelled commands.  
%We use the Jaro similarity metric~\cite{jaro:89}, as it is a widespread distance measure that was introduced to capture typographical errors.
%Querying alias names with the same length as command names, but differ in their content (i.e., they do not define default arguments) returns \num{518500} (\per{10.86}) results.
We query alias names with the same length as command names, but differ in their content (i.e., they do not define default arguments), apply the Damerau-Levenshtein algorithm, and normalize the result.
Normalized distances are a float in the range of 0 and 1, where 0 entails there are no differences between the two strings and 1 means the distance is as far as it can be (corresponding to the length of the longer string in the original score).
We compute the results, order by normalized score, and empirically determine $\frac{1}{3}$ (roughly corresponding to \per{33} difference between autocorrecting alias name and command name) to be considered a conservative threshold to determine misspellings.
This results in \num{63875} (\per{1.3}) total aliases that serve as autocorrect rules.
Table \ref{tab:autocorrect} provides an overview of interesting autocorrecting aliases with different distance measures.

\begin{table}[]
	\caption{A curated selection from autocorrecting aliases highlighting a variety of use cases together with their normalized Damerau-Levenshtein distance}
		\label{tab:autocorrect}
	\begin{tabular}{@{}llll@{}}
		\toprule
		\textbf{Commands}                                                      & \textbf{Alias}                                                       & \textbf{Reason}      & \textbf{Score} \\ \midrule
		grep, sudo                                                             & grpe, sduo                                                           & Transposition        & 0.25           \\ \hline
		pluralise                                                              & pluralize                                                            & Localized Spelling   & 0.11           \\ \hline
		\begin{tabular}[c]{@{}l@{}}docker\_build, \\ mysql-server\end{tabular} & \begin{tabular}[c]{@{}l@{}}docker-build,\\ mysql.server\end{tabular} & Punctuation          & 0.08           \\ \hline
		jupyter                                                                & Jupyter                                                              & Case-sensitivity     & 0.14           \\ \hline
		gunzip                                                                 & ungzip                                                               & Mismatched casing & 0.33           \\ \bottomrule
	\end{tabular}
\end{table}



\subsection{Compression}

\begin{figure}
    \centering
    \includegraphics[width=\columnwidth]{compression_dist.pdf}
    \caption{Distribution of alias compression ratios}
    \label{fig:compression}
\end{figure}

One use of aliases is to simplify complex expressions by giving them short, memorable names.
The average length of an alias name is \num{4.43} characters, whereas the average length of an alias value is \num{21.15} characters.
If we divide the length of an alias value by the length of the alias name, we get the \emph{compression ratio} of the alias.
For example, the definition
\begin{CVerbatim}
alias gs='git status'
\end{CVerbatim}
has a compression ratio of 5.
\Cref{fig:compression} shows the distribution of compression ratios over all aliases in the dataset.
The median compression ratio is 4.2, meaning half of all alias values are at least four times as long as their alias names.
A compression ratio less than 1 indicates an alias name that is longer than the value it aliases.
\Cref{tab:use-cases} gives at-a-glance compression distributions for aliases using common commands in first position.
The mini-histograms are log-log scaled; red lines mark the compression ratio 1.

There are \num{157542} aliases (\per{3.3}) with names longer than their values.%
\footnote{The curious spike at 0.36 in \cref{fig:compression} is the \texttt{zsh} portability hack described earlier.
It makes up \per{49.68} of long alias names.}
Among them, we found many descriptive names, like
\begin{itemize}
    \item \verb|git-last-commit-message|
    \item \verb|docker_list_all_containers|
    \item \verb|generate_random_password|
\end{itemize}
and similar wordy descriptions of simple command invocations.
Additionally, we found compatibility aliases (like \texttt{md5sum} for \texttt{md5}) and many humorous definitions (like \texttt{kthxbai} for \texttt{halt} or \texttt{please} for \texttt{sudo}).

The two longest alias names we found are from joke definitions.
The first is \num{1772} characters long and is comprised of the letter `f' repeated \num{1053} times, followed by the letter `u' repeated \num{719} times.
It is an alias for the \texttt{cat} command with a similarly named file as an argument.
The second longest alias name is a Swedish compound word of \num{131} characters,\footnote{Translating, roughly, to northwestern-glacier-artillery-flight-thrust-simulator-plant-equipment-maintenance-follow-up-systems-discussion-posts-preparation-works.} aliasing the \texttt{ls} command.

On the other end of the spectrum, the highest compression ratio in any alias definition comes from an alias named \texttt{BEEP}, which invokes the Linux \texttt{beep} utility 9 times in succession, with a combined \num{4471} arguments.
When executed, it appears to play Daft Punk's 2001 instrumental single \emph{Aerodynamic}.


\subsection{Chaining}

One source of complexity (i.e. value length) in aliases is command composition or chaining.
\num{351060} aliases (\per{7.36}) are composed of more than one command.
the most popular way to combine commands is with a pipe (\verb!|!), used by \per{35.2} of multi-command aliases, closely followed by logical conjunction (\verb|&&|), used by \per{29.75}, and simple chaining (\verb|;|), used by \per{28.4}.
Other operators (\verb|&|, \verb!||!, \verb!|&!) appear in only \per{6.65} of multi-command aliases.

Table \TODO shows the 


The pipe operator is the  most interesting, as it lets ascertain data flow between commands.
Figure \TODO visualizes ..


\subsection{Safety and Interactive Mode}
\label{sec:safety}

The \texttt{sudo} command allows the user to execute another command with superuser privileges.
In our dataset, we found \num{175806} commands (\per{6.75}) prefixed with \texttt{sudo}, used in \num{158779} alias definitions (\per{6.8}).
The top 10 commands used with \texttt{sudo} are given in \Cref{tab:sudo-commands}.

Remarkably, 9 out of the top 10 \texttt{sudo}-prefixed commands are related to package managers:
\texttt{zypper} (openSUSE); \texttt{pacman}, \texttt{abs} and \texttt{aur} (Arch Linux); \verb|apt-get| and \verb|$apt-pref| (Debian, Ubuntu); \texttt{yum} (RPM); \texttt{dnf} (Fedora); and \texttt{port} (macOS).
Even more remarkable: all of these commands are used with \texttt{sudo} the majority of the time they are used.
Some users even go so far as to redefine certain commands to always be executed with \texttt{sudo}, mostly system utilities like \texttt{reboot} or \texttt{shutdown}, but also (again) package managers.% like \texttt{pacman} and \texttt{apt}.
%However, on the whole this practice appears to be relatively rare (see \cref{tab:sudo-redefine}).

\begin{table}
    \caption{Top 10 commands used with \texttt{sudo}.}
    \label{tab:sudo-commands}
    \begin{tabular}{lrrl}
      \toprule
      Command & \# & \multicolumn{1}{c}{With \texttt{sudo}} & Description \\
      \midrule  
      \verb|zypper|    & \num{33753} & \num{31493} (\per{93.30})  & package manager\\
      \verb|pacman|    & \num{25322} & \num{17422} (\per{68.80})  & package manager \\
      \verb|apt-get|   & \num{16562} & \num{13946} (\per{84.20})  & package manager \\
      \verb|$apt_pref| & \num{10464} & \num{10464} (\per{100.00}) & package manager \\
      \verb|dnf|       & \num{12302} &  \num{6459} (\per{52.50})  & package manager \\
      \verb|yum|       & \num{11856} &  \num{6428} (\per{54.22})  & package manager \\
      \verb|port|      & \num{5574} &   \num{5416} (\per{97.17})  & package manager \\
      \verb|abs|       & \num{5097} &   \num{5094} (\per{99.94})  & package manager \\
      \verb|aur|       & \num{4525} &   \num{4489} (\per{99.20})  & package manager \\
      \verb|systemctl| & \num{7399} &   \num{3855} (\per{52.10})  & system utility \\
      \bottomrule
    \end{tabular}
\end{table}

On the other end of the safety spectrum, some users default to running the file system commands \texttt{rm}, \texttt{cp}, and \texttt{mv} in interactive mode, which prompts before performing potentially destructive actions.
Taking into account variations between systems and different ways to enable interactive mode, \per{36.83} of alias definitions invoking \texttt{rm} are redefinitions of \texttt{rm} enabling interactive mode by default.
Those numbers are even higher for \texttt{cp} and \texttt{mv}, where \per{65.97} and \per{74.49} of uses are redefinitions, respectively (see \Cref{tab:interactive}).

\begin{table}
    \caption{Commands commonly used in interactive mode}
    \label{tab:interactive}
    \begin{tabular}{lrrr}
        \toprule
        Command & \# & \multicolumn{1}{c}{With \texttt{-i}} & \multicolumn{1}{c}{Redefined With \texttt{-i}} \\
        \midrule
        rm & \num{20451} & \num{7878} (\per{38.52}) & \num{7533} (\per{36.83}) \\
        cp & \num{11573} & \num{7755} (\per{67.00}) & \num{7635} (\per{65.97}) \\
        mv & \num{9328} & \num{7068} (\per{75.77}) & \num{6948} (\per{74.49}) \\
        \bottomrule
    \end{tabular}
\end{table}


\subsection{Bookmarks}

We define a \emph{bookmark} to be an alias containing an argument that references some specific local or remote location, e.g. a file path, domain or IP address.
For instance, 
\begin{CVerbatim}
alias dl="cd ~/Downloads"
\end{CVerbatim}
and
\begin{CVerbatim}
alias starwars="telnet towel.blinkenlights.nl"
\end{CVerbatim}
are both bookmark aliases.

To find such bookmarks in our dataset, we searched for arguments that are locations, which we take to be any of the following:
\begin{itemize}
    \item An IPv4 address, matched by the liberal regular expression \verb|[0-9]+.[0-9]+.[0-9]+.[0-9]+|
    \item A string containing one of the known top-level domains\footnote{From \url{http://data.iana.org/TLD/tlds-alpha-by-domain.txt} (retrieved January 14, 2020)} preceded by a dot (\verb|.|) and followed by a slash (\verb|/|), colon (\verb|:|) or the end of the string.
    \item A string containing a forward slash (\verb|/|), indicating a path.
\end{itemize}
To avoid false positives, we sampled the top 300 search results according to the above criteria and determined some exclusion patterns.
For instance, \texttt{/dev/null} is not a location for our purposes.
Neither is \texttt{origin/master}, and thus 
\begin{CVerbatim}
alias gm="git merge origin/master"
\end{CVerbatim}
does not count as a bookmark.

By our definition, \num{557252} aliases (\per{11.68}) are bookmarks.
Of these, only \num{75169} are remote bookmarks (containing URLs or IP addresses).
Bookmarks are used predominantly for file system navigation, and the \texttt{cd} command is featured heavily in bookmark aliases.
Most other uses seem to be development related, like starting services such as web servers or databases with pre-defined locations, or outputting log files, as in 
\begin{CVerbatim}
alias onoz="cat /var/log/errors.log"
\end{CVerbatim}
or opening frequently edited files, as in the most common bookmark overall, which is for editing the shell configuration itself:
\begin{CVerbatim}
alias ohmyzsh="mate ~/.oh-my-zsh"
\end{CVerbatim}

In \cref{tab:use-cases} we have marked commands 



% for potential related work (including method), google 'usability smells'
% e.g. https://www.sciencedirect.com/science/article/pii/S1071581916301215
%      https://link.springer.com/chapter/10.1007/978-3-662-44811-3_13