\subsection{Safety and Interactive Mode}
\label{sec:safety}

The \texttt{sudo} command allows the user to execute another command with superuser privileges.
In our dataset, we found \num{336298} commands (\per{6.3}) prefixed with \texttt{sudo}, used in \num{305637} alias definitions (\per{6.4}).
The top 10 commands used with \texttt{sudo} are given in \Cref{tab:sudo-commands}.

Remarkably, 8 out of the top 10 \texttt{sudo}-prefixed commands are related to package managers:
\texttt{zypper} (openSUSE); \texttt{pacman}, \texttt{abs} and \texttt{aur} (Arch Linux); \verb|apt-get| and \verb|$apt-pref| (Debian, Ubuntu); \texttt{yum} (RPM); and \texttt{dnf} (Fedora).
Even more remarkable: all of these commands are used with \texttt{sudo} the majority of the time they are used.
Some users even go so far as to redefine certain commands to always be executed with \texttt{sudo}, mostly system utilities like \texttt{reboot} or \texttt{shutdown}, but also (again) package managers like \texttt{pacman} and \texttt{apt}.
However, on the whole this practice appears to be relatively rare (see \cref{tab:sudo-redefine}).

\begin{table}
    \caption{Top 10 commands used with \texttt{sudo}.}
    \label{tab:sudo-commands}
    \begin{tabular}{lrrl}
      \toprule
      Command & \# & \multicolumn{1}{c}{With \texttt{sudo}} & Description \\
      \midrule  
      \verb|zypper|    & \num{66295} & \num{61891} (\per{93.36})  & package manager\\
      \verb|pacman|    & \num{46821} & \num{32407} (\per{69.21})  & package manager \\
      \verb|apt-get|   & \num{34075} & \num{28323} (\per{83.12})  & package manager \\
      \verb|$apt_pref| & \num{20245} & \num{20245} (\per{100.00}) & package manager \\
      \verb|yum|       & \num{23940} & \num{12843} (\per{53.65})  & package manager \\
      \verb|dnf|       & \num{24469} & \num{12697} (\per{51.89})  & package manager \\
      \verb|port|      & \num{11228} & \num{10863} (\per{96.75})  & networking utility \\
      \verb|abs|       &  \num{9909} &  \num{9905} (\per{99.96})  & package manager \\
      \verb|aur|       &  \num{8837} &  \num{8795} (\per{99.52})  & package manager \\
      \verb|systemctl| & \num{13942} &  \num{7018} (\per{50.34})  & system utility \\
      \bottomrule
    \end{tabular}
\end{table}

\begin{table}
    \caption{Top 10 commands redefined to always use \texttt{sudo}}
    \label{tab:sudo-redefine}
    \begin{tabular}{llrl}
        \toprule
        Alias Name & Alias Value & \# & Description \\
        \midrule
        \verb|pacman|    & \verb|sudo pacman|    & 901 & package manager \\
        \verb|reboot|    & \verb|sudo reboot|    & 809 & system utility \\
        \verb|shutdown|  & \verb|sudo shutdown|  & 499 & system utility \\
        \verb|apt-get|   & \verb|sudo apt-get|   & 455 & package manager \\
        \verb|docker|    & \verb|sudo docker|    & 345 & container platform \\
        \verb|halt|      & \verb|sudo halt|      & 230 & system utility \\
        \verb|apt|       & \verb|sudo apt|       & 228 & package manager \\
        \verb|systemctl| & \verb|sudo systemctl| & 184 & system utility \\
        \verb|htop|      & \verb|sudo htop|      & 180 & system utility \\
        \verb|poweroff|  & \verb|sudo poweroff|  & 169 & system utility \\
        \bottomrule
    \end{tabular}
\end{table}

On the other end of the safety spectrum, some users default to running the file system commands \texttt{rm}, \texttt{cp}, and \texttt{mv} in interactive mode, which prompts before performing potentially destructive actions.
Taking into account variations between systems and different ways to enable interactive mode, \per{44.37} of alias definitions invoking \texttt{rm} are redefinitions of \texttt{rm} enabling interactive mode by default.
Those numbers are even higher for \texttt{cp} and \texttt{mv}, where \per{68.27} and \per{77.62} of uses are redefinitions, respectively (see \Cref{tab:interactive}).

\begin{table}
    \caption{Commands commonly used in interactive mode}
    \label{tab:interactive}
    \begin{tabular}{lrrr}
        \toprule
        Command & \# & \multicolumn{1}{c}{With \texttt{-i}} & \multicolumn{1}{c}{Redefined With \texttt{-i}} \\
        \midrule
        rm & \num{44209} & \num{20300} (\per{45.92}) & \num{19617} (\per{44.37}) \\
        cp & \num{27472} & \num{18994} (\per{69.14}) & \num{18755} (\per{68.27}) \\
        mv & \num{22689} & \num{17842} (\per{78.64}) & \num{17611} (\per{77.62}) \\
        \bottomrule
    \end{tabular}
\end{table}
