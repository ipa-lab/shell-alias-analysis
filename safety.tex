\subsection{Safety and Interactive Mode}
\label{sec:safety}

The \texttt{sudo} command allows the user to execute another command with superuser privileges.
In our dataset, we found \num{175806} commands (\per{6.75}) prefixed with \texttt{sudo}, used in \num{158779} alias definitions (\per{6.8}).
The top 10 commands used with \texttt{sudo} are given in \Cref{tab:sudo-commands}.

Remarkably, 9 out of the top 10 \texttt{sudo}-prefixed commands are related to package managers:
\texttt{zypper} (openSUSE); \texttt{pacman}, \texttt{abs} and \texttt{aur} (Arch Linux); \verb|apt-get| and \verb|$apt-pref| (Debian, Ubuntu); \texttt{yum} (RPM); \texttt{dnf} (Fedora); and \texttt{port} (macOS).
Even more remarkable: all of these commands are used with \texttt{sudo} the majority of the time they are used.
Some users even go so far as to redefine certain commands to always be executed with \texttt{sudo}, mostly system utilities like \texttt{reboot} or \texttt{shutdown}, but also (again) package managers.% like \texttt{pacman} and \texttt{apt}.
%However, on the whole this practice appears to be relatively rare (see \cref{tab:sudo-redefine}).

\begin{table}
    \caption{Top 10 commands used with \texttt{sudo}.}
    \label{tab:sudo-commands}
    \begin{tabular}{lrrl}
      \toprule
      Command & \# & \multicolumn{1}{c}{With \texttt{sudo}} & Description \\
      \midrule  
      \verb|zypper|    & \num{33753} & \num{31493} (\per{93.30})  & package manager\\
      \verb|pacman|    & \num{25322} & \num{17422} (\per{68.80})  & package manager \\
      \verb|apt-get|   & \num{16562} & \num{13946} (\per{84.20})  & package manager \\
      \verb|$apt_pref| & \num{10464} & \num{10464} (\per{100.00}) & package manager \\
      \verb|dnf|       & \num{12302} &  \num{6459} (\per{52.50})  & package manager \\
      \verb|yum|       & \num{11856} &  \num{6428} (\per{54.22})  & package manager \\
      \verb|port|      & \num{5574} &   \num{5416} (\per{97.17})  & package manager \\
      \verb|abs|       & \num{5097} &   \num{5094} (\per{99.94})  & package manager \\
      \verb|aur|       & \num{4525} &   \num{4489} (\per{99.20})  & package manager \\
      \verb|systemctl| & \num{7399} &   \num{3855} (\per{52.10})  & system utility \\
      \bottomrule
    \end{tabular}
\end{table}

On the other end of the safety spectrum, some users default to running the file system commands \texttt{rm}, \texttt{cp}, and \texttt{mv} in interactive mode, which prompts before performing potentially destructive actions.
Taking into account variations between systems and different ways to enable interactive mode, \per{36.83} of alias definitions invoking \texttt{rm} are redefinitions of \texttt{rm} enabling interactive mode by default.
Those numbers are even higher for \texttt{cp} and \texttt{mv}, where \per{65.97} and \per{74.49} of uses are redefinitions, respectively (see \Cref{tab:interactive}).

\begin{table}
    \caption{Commands commonly used in interactive mode}
    \label{tab:interactive}
    \begin{tabular}{lrrr}
        \toprule
        Command & \# & \multicolumn{1}{c}{With \texttt{-i}} & \multicolumn{1}{c}{Redefined With \texttt{-i}} \\
        \midrule
        rm & \num{20451} & \num{7878} (\per{38.52}) & \num{7533} (\per{36.83}) \\
        cp & \num{11573} & \num{7755} (\per{67.00}) & \num{7635} (\per{65.97}) \\
        mv & \num{9328} & \num{7068} (\per{75.77}) & \num{6948} (\per{74.49}) \\
        \bottomrule
    \end{tabular}
\end{table}
