Shell aliases allow command line users to define command substitutions.
When processing the command line, the shell replaces each alias name with its string value, which is in turn executed.
%For instance, the common alias \verb|ll| may be replaced with \verb|ls -l|.
The definitions can take the form of any arbitrarily complex command or chain of commands.
To characterize patterns and understand the complexities of alias usage, we conduct an exploratory empirical study on over 2.3 million alias definitions found on GitHub.
We find that aliases have multiple complex use cases: enforcing default arguments, enabling safety and interaction, storing bookmarks to local and remote locations, as well as providing autocorrect for misspelled commands.
When analyzing aliases with respect to compression, they were not only used to abbreviate and simplify longer and complicated command and argument structures, but also to lend them more descriptive (and longer) names.
We conjecture that the deeper understanding of aliasing practices in command lines enabled by this study can support tool developers in designing better user experiences for command line programs in general.
More specifically, it can point to particular usability issues within the command line programs referenced in our analysis and dataset.