\section{Threats to Validity}

We review potential limitations of our study as threats to validity.
First, our sample might not be representative.
Our dataset only includes aliases by people who publicly shared their dotfiles, we only collected from GitHub, and our sample does not include forks.
Nevertheless, our dataset is very exhaustive, as we were able to sample \per{94.09} of the estimated population of Shell files containing aliases on GitHub.
And while mining GitHub can be fraught with perils \cite{kalliamvakou:14}, we specifically sought out personal repositories, side-stepping many of the typical issues with mining GitHub for software projects.

Second, our parser might not be sophisticated enough to recognize complex real-world aliases or cope with minute platform differences.
To mitigate this threat, we ran multiple sanity checks and tested the parser on some hairy examples from the dataset.
We did not detect any significant mis-parses and think that we have covered the majority of relevant cases.
The raw unparsed database is available in our replication package.

Third, aliases might not reflect intent as much as we assume.
En-masse copy-pasting of aliases by users, without them knowing exactly what they are copying, is certainly a realistic scenario.
System distributions and configuration frameworks like \emph{ohmyzsh} ship with numerous aliases by default or as part of easily enabled plugins.
Users might not even be aware of the aliases they have on their system.
This is mitigated by the fact that all of the aliases we collected were publicly shared by users.
It stands to reason that even if a user is not aware of all the details of their system configuration, they confirm their attachment to this configuration and its aliases by publicly sharing them---even if only for the purposes of synchronizing them across the users' own machines.

Fourth, we might not actually be able to see the true user intent, if it exists, as quantitative measures might hide a long tail of minor variations and individual user preference.
Conclusions about common aliases or selected subsets might not be generalizable.
To mitigate these summarizing effects, we established use cases as a vehicle to take a deeper dive into the details of certain alias usage.
Since we sampled almost the whole available population, we are confident in the strength of our data and the conclusions we can draw from particular instances.
Our replication package includes our whole toolchain and all alias data in a relational format ready for further analysis.
