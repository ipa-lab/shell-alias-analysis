The single most common alias definition in our dataset is
\begin{verbatim}
alias '${1+"$@"}'="$@"
\end{verbatim}
appearing \numprint{78265} times, which is \numprint[\%]{1.64} of all alias definitions.
This curious jumble of symbols derives from a portability hack for \texttt{zsh}.\footnote{\url{https://unix.stackexchange.com/a/191836}}

More familiar results can be seen in \Cref{tab:top-summary}, which shows the most common alias names, commands, and arguments appearing in alias definitions.
The most common alias name we found is \texttt{ls}, appearing a total number of \numprint{124515} times, which is \numprint[\%]{2.61} of all alias definitions.
Note that this is \texttt{ls} as an \emph{alias name}, a redefinition of the \texttt{ls} \emph{command}, which appears \numprint{378097} times (\numprint[\%]{7.12}).
This is about half as often as \texttt{git}, the most common command, which appears in \numprint{626185} aliases (\numprint[\%]{11.79}).
The most common argument, across all commands, is \texttt{--color=auto}, appearing \numprint{268171} times (\numprint[\%]{3.6})

\begin{table*}
    \caption{Top alias names, commands and arguments}
    \label{tab:top-summary}
    \begin{tabular}{lrr}
        \toprule
                 Alias Name &                 \# &               \% \\
        \midrule
                  \verb|ls| &  \numprint{124515} &  \numprint{2.61} \\
                  \verb|ll| &   \numprint{89254} &  \numprint{1.87} \\
         \verb|'${1+"$@"}'| &   \numprint{78265} &  \numprint{1.64} \\
                \verb|grep| &   \numprint{63392} &  \numprint{1.33} \\
                  \verb|la| &   \numprint{62039} &  \numprint{1.30} \\
        \bottomrule
    \end{tabular}
    \hspace{0.1cm}
    \begin{tabular}{lrr}
        \toprule
               Command &                 \# &                \% \\
        \midrule
            \verb|git| &  \numprint{626185} &  \numprint{11.79} \\
             \verb|ls| &  \numprint{378097} &   \numprint{7.12} \\
             \verb|cd| &  \numprint{229522} &   \numprint{4.32} \\
           \verb|grep| &  \numprint{128045} &   \numprint{2.41} \\
           \verb|"$@"| &   \numprint{78276} &   \numprint{1.47} \\
        \bottomrule
    \end{tabular}
    \hspace{0.1cm}
    \begin{tabular}{lrr}
        \toprule
                    Argument &                 \# &               \% \\
        \midrule
         \verb|--color=auto| &  \numprint{268171} &  \numprint{3.60} \\
                   \verb|-i| &  \numprint{109094} &  \numprint{1.47} \\
                   \verb|-v| &   \numprint{70424} &  \numprint{0.95} \\
                   \verb|-l| &   \numprint{63302} &  \numprint{0.85} \\
                   \verb|-a| &   \numprint{62023} &  \numprint{0.83} \\
        \bottomrule
    \end{tabular}        
\end{table*}

Looking at each part of an alias definition in isolation can only get us so far, as arguments only gain meaning in conjunction with commands and alias names potentially vary widely.
\Cref{tab:command-summary} gives a more informative view, centered around some common commands, showing us the top arguments given with each command and the most common alias names by which the command/argument combinations are referred to.
Here we can see \TODO

\begin{table*}
    \caption{Common commands and their top arguments and aliases}
    \label{tab:command-summary}
    \begin{tabular}{lrll}
        \toprule
             Command &                \% &                 Arguments &                                                                                 Aliases (\%) \\
        \midrule
          \verb|git| &   \numprint{3.75} &             \verb|status| &                                  \verb|gs| (\numprint{51.57}), \verb|gst| (\numprint{24.17}) \\
                     &   \numprint{2.42} &           \verb|checkout| & \verb|gco| (\numprint{51.17}), \verb|gc| (\numprint{10.04}), \verb|letcat| (\numprint{9.18}) \\
                     &   \numprint{2.33} &               \verb|push| &    \verb|gp| (\numprint{49.12}), \verb|rulz| (\numprint{9.53}), \verb|gps| (\numprint{7.15}) \\
                     &   \numprint{2.31} &                   \verb|| &                                                                  \verb|g| (\numprint{78.87}) \\
                     &   \numprint{2.18} &               \verb|pull| &    \verb|gl| (\numprint{25.32}), \verb|gpl| (\numprint{13.29}), \verb|gp| (\numprint{11.53}) \\
                     &   \numprint{2.11} &               \verb|diff| &                                                                 \verb|gd| (\numprint{83.21}) \\
                     &   \numprint{2.05} &                \verb|add| &                              \verb|ga| (\numprint{73.01}), \verb|chicken| (\numprint{10.85}) \\
                     &   \numprint{2.01} &             \verb|branch| &                                   \verb|gb| (\numprint{74.88}), \verb|gbr| (\numprint{8.26}) \\
                     &   \numprint{1.59} &          \verb|commit -m| & \verb|gcm| (\numprint{21.67}), \verb|gcmsg| (\numprint{19.59}), \verb|gc| (\numprint{18.18}) \\
                     &   \numprint{1.20} &             \verb|commit| &                                                                 \verb|gc| (\numprint{65.51}) \\
        \midrule
           \verb|ls| &  \numprint{13.15} &       \verb|--color=auto| &                                                                 \verb|ls| (\numprint{99.09}) \\
                     &   \numprint{9.12} &                 \verb|-A| &                                                                 \verb|la| (\numprint{98.27}) \\
                     &   \numprint{8.57} &                \verb|-CF| &                                                                  \verb|l| (\numprint{98.94}) \\
                     &   \numprint{6.69} &                 \verb|-l| &                                                                 \verb|ll| (\numprint{82.64}) \\
                     &   \numprint{6.63} &               \verb|-alF| &                                                                 \verb|ll| (\numprint{97.89}) \\
                     &   \numprint{3.65} &                   \verb|| &      \verb|l| (\numprint{21.38}), \verb|sl| (\numprint{16.99}), \verb|iz| (\numprint{12.24}) \\
                     &   \numprint{3.06} &                 \verb|-G| &                                                                 \verb|ls| (\numprint{97.46}) \\
                     &   \numprint{2.43} &                \verb|-la| &      \verb|ll| (\numprint{32.80}), \verb|la| (\numprint{22.61}), \verb|l| (\numprint{12.71}) \\
                     &   \numprint{2.01} &                 \verb|-a| &                                                                 \verb|la| (\numprint{74.80}) \\
                     &   \numprint{1.77} &               \verb|-lah| &     \verb|l| (\numprint{32.64}), \verb|lsa| (\numprint{31.61}), \verb|ll| (\numprint{19.01}) \\
        \midrule
         \verb|grep| &  \numprint{36.12} &       \verb|--color=auto| &                                                               \verb|grep| (\numprint{99.29}) \\
                     &  \numprint{12.50} &                   \verb|| &   \verb|G| (\numprint{22.27}), \verb|hgrep| (\numprint{10.98}), \verb|rrg| (\numprint{7.22}) \\
                     &   \numprint{4.42} &            \verb|--color| &                                                               \verb|grep| (\numprint{97.00}) \\
                     &   \numprint{1.68} &                 \verb|-i| &        \verb|grep| (\numprint{6.88}), \verb|G| (\numprint{5.86}), \verb|g| (\numprint{5.86}) \\
                     &   \numprint{1.52} & \verb|--color=never '^d'| &                                                                \verb|lsd| (\numprint{98.45}) \\
        \bottomrule
    \end{tabular}
\end{table*}

One drawback of this presentation is that there are obviously many different ways of how an arguments to a command can be combined, since for most programs, the order of arguments does not matter and often there are multiple ways to specify the same argument.
\TODO
